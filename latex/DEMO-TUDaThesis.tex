\documentclass[
	ngerman,
	ruledheaders=section,%Ebene bis zu der die Überschriften mit Linien abgetrennt werden, vgl. DEMO-TUDaPub
	class=report,% Basisdokumentenklasse. Wählt die Korrespondierende KOMA-Script Klasse
	thesis={type=master},% Dokumententyp Thesis, für Dissertationen siehe die Demo-Datei DEMO-TUDaPhd
	accentcolor=3c,% Auswahl der Akzentfarbe
	custommargins=true,% Ränder werden mithilfe von typearea automatisch berechnet
	marginpar=false,% Kopfzeile und Fußzeile erstrecken sich nicht über die Randnotizspalte
	%BCOR=5mm,%Bindekorrektur, falls notwendig
	parskip=half-,%Absatzkennzeichnung durch Abstand vgl. KOMA-Sript
	fontsize=11pt,%Basisschriftgröße laut Corporate Design ist mit 9pt häufig zu klein
%	logofile=example-image, %Falls die Logo Dateien nicht vorliegen
]{tudapub}

\usepackage[acronym,nonumberlist,nopostdot,shortcuts]{glossaries}

\usepackage{url}
\usepackage{graphicx}
\usepackage{tikz}
\usepackage{graphicx}
\usepackage{xfrac}
%\usepackage{amssymb}
\usepackage{epstopdf} 
\usepackage{eurosym}
%\usepackage{lmodern} % nicht nutzen dies schaltet die TU Schriftart aus!!!!!!!!!
\usepackage{subfig}
\usepackage{booktabs} % hübschere Tabellen als der LaTeX-Standard
\usepackage{amsmath}
\usepackage{placeins}
\usepackage{siunitx}
\usepackage{multirow}
\usepackage{adjustbox}
\usepackage{mwe}
\usepackage{subfig}
\usepackage{paralist}
% Der folgende Block ist nur bei pdfTeX auf Versionen vor April 2018 notwendig
\usepackage{iftex}
\ifPDFTeX
\usepackage[utf8]{inputenc}%kompatibilität mit TeX Versionen vor April 2018
\fi

%%%%%%%%%%%%%%%%%%%
%Sprachanpassung & Verbesserte Trennregeln
%%%%%%%%%%%%%%%%%%%
\usepackage[english, main=english]{babel}
\usepackage[autostyle]{csquotes}% Anführungszeichen vereinfacht
\usepackage{microtype}


%%%%%%%%%%%%%%%%%%%
%Literaturverzeichnis
%%%%%%%%%%%%%%%%%%%
\usepackage{biblatex}   % Literaturverzeichnis
%\bibliography{DEMO-TUDaBibliography}


%%%%%%%%%%%%%%%%%%%
%Tabellen
%%%%%%%%%%%%%%%%%%%
%\usepackage{array}     % Basispaket für Tabellenkonfiguration, wird von den folgenden automatisch geladen
\usepackage{tabularx}   % Tabellen, die sich automatisch der Breite anpassen
%\usepackage{longtable} % Mehrseitige Tabellen
%\usepackage{xltabular} % Mehrseitige Tabellen mit anpassarer Breite
\usepackage{booktabs}   % Verbesserte Möglichkeiten für Tabellenlayout über horizontale Linien

%%%%%%%%%%%%%%%%%%%
%Paketvorschläge Mathematik
%%%%%%%%%%%%%%%%%%%
\usepackage{mathtools} % erweiterte Fassung von amsmath
\usepackage{amssymb}   % erweiterter Zeichensatz
\usepackage{siunitx}   % Einheiten



%Formatierungen für Beispiele in diesem Dokument. Im Allgemeinen nicht notwendig!
\let\file\texttt
\let\code\texttt

\usepackage{pifont}% Zapf-Dingbats Symbole
\newcommand*{\FeatureTrue}{\ding{52}}
\newcommand*{\FeatureFalse}{\ding{56}}
%\bibliographystyle{IEEEtran}



\bibliography{bib/Literatur}
%Own Commands



\linespread{1.25}





\begin{document}

\Metadata{
	title=Modeling and Analysis of Human Behavior Impacts on Energy Systems during Crisis Events,
	author=Isabella Nunes Grieser
}

\title{Modeling and Analysis of Human Behavior Impacts on 
Energy Systems during Crisis Events}
\subtitle{Modellierung und Analyse der Auswirkung des menschlichen
Verhaltens auf Energienetze während Krisensituationen}
\author[short Name]{Isabella Nunes Grieser}%optionales Argument ist die Signatur, 
%\birthplace{Geburtsort}%Geburtsort, bei Dissertationen zwingend notwendig
\reviewer{Prof. Dr. Florian Steinke \and second supervisor}%Gutachter
\publishers{}
%Diese Felder erden untereinander auf der Titelseite platziert. 
%\department ist eine notwendige Angabe, siehe auch dem Abschnitt `Abweichung von den Vorgaben für die Titelseite'
\department{etit} % Das Kürzel wird automatisch ersetzt und als Studienfach gewählt, siehe Liste der Kürzel im Dokument.
\institute{\includegraphics*[width=4cm]{gfx/EINS_Logo}}
\group{}

\submissiondate{\today}
\examdate{\today}

%	\tuprints{urn=1234,printid=12345}
%	\dedication{Für alle, die \TeX{} nutzen.}

\maketitle

\affidavit

\tableofcontents


\listoftables
\listoffigures
\printglossary %[type = \acronymtype]

\chapter{Introduction}
An Introduction should contain the theoretical background, the problem adressed, the problem's implication in science and/or society. It should than explain a short summary of the state of the art  as well as the composition of the thesis.
\chapter{Theory}
The theoretical framework should be stated herein
\chapter{Experiments}
Experiments should be carried out to verify or falsify the theory developed earlier. At EINS, this usually involves your simulations, which should be explained.
\chapter{Results}

The experiments outcome should be critically discussed. A conclusion about the theory based on the experiments follows. Failures and shortcomings of the theoretical model should be pointed out and explained as necessary.
\chapter{Outlook}
A short summary of the thesis follows an contains what was the problem under focus, how was it tackled and what was achieved in the thesis. At last, further suggestions for future research shows the comprehension of the own work as well as the scientific procedure of ever refining a theory till it becomes a model.

%\appendix

%\input{inc/diaApx.tex}


		
\printbibliography
	
\end{document}
