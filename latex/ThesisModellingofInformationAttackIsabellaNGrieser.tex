\documentclass[
	ngerman,
	ruledheaders=section,%Ebene bis zu der die Überschriften mit Linien abgetrennt werden, vgl. DEMO-TUDaPub
	class=report,% Basisdokumentenklasse. Wählt die Korrespondierende KOMA-Script Klasse
	thesis={type=master},% Dokumententyp Thesis, für Dissertationen siehe die Demo-Datei DEMO-TUDaPhd
	accentcolor=3c,% Auswahl der Akzentfarbe
	custommargins=false,% Ränder werden mithilfe von typearea automatisch berechnet
	marginpar=false,% Kopfzeile und Fußzeile erstrecken sich nicht über die Randnotizspalte
	%BCOR=5mm,%Bindekorrektur, falls notwendig
	parskip=half-,%Absatzkennzeichnung durch Abstand vgl. KOMA-Sript
	fontsize=11pt,%Basisschriftgröße laut Corporate Design ist mit 9pt häufig zu klein
%	logofile=example-image, %Falls die Logo Dateien nicht vorliegen
	IMRAD=false,%Abschalten von IMRAD-Warnings wegen fehlender Labels
]{tudapub}

\usepackage[acronym,nonumberlist,nopostdot,shortcuts]{glossaries}

\usepackage{url}
\usepackage{graphicx}
\usepackage{tikz}
\usepackage{graphicx}
\usepackage{xfrac}
%\usepackage{amssymb}
\usepackage{epstopdf} 
\usepackage{eurosym}
%\usepackage{lmodern} % nicht nutzen dies schaltet die TU Schriftart aus!!!!!!!!!
\usepackage{subfig}
\usepackage{booktabs} % hübschere Tabellen als der LaTeX-Standard
\usepackage{amsmath}
\usepackage{placeins}
\usepackage{siunitx}
\usepackage{multirow}
\usepackage{adjustbox}
\usepackage{mwe}
\usepackage{subfig}
\usepackage{paralist}
% Der folgende Block ist nur bei pdfTeX auf Versionen vor April 2018 notwendig
\usepackage{iftex}
\ifPDFTeX
\usepackage[utf8]{inputenc}%kompatibilität mit TeX Versionen vor April 2018
\fi

%%%%%%%%%%%%%%%%%%%
%Sprachanpassung & Verbesserte Trennregeln
%%%%%%%%%%%%%%%%%%%
\usepackage[english, main=english]{babel}
\usepackage[autostyle]{csquotes}% Anführungszeichen vereinfacht
\usepackage{microtype}


%%%%%%%%%%%%%%%%%%%
%Literaturverzeichnis
%%%%%%%%%%%%%%%%%%%
\usepackage{biblatex}   % Literaturverzeichnis
%\bibliography{DEMO-TUDaBibliography}


%%%%%%%%%%%%%%%%%%%
%Tabellen
%%%%%%%%%%%%%%%%%%%
%\usepackage{array}     % Basispaket für Tabellenkonfiguration, wird von den folgenden automatisch geladen
\usepackage{tabularx}   % Tabellen, die sich automatisch der Breite anpassen
%\usepackage{longtable} % Mehrseitige Tabellen
%\usepackage{xltabular} % Mehrseitige Tabellen mit anpassarer Breite
\usepackage{booktabs}   % Verbesserte Möglichkeiten für Tabellenlayout über horizontale Linien

%%%%%%%%%%%%%%%%%%%
%Paketvorschläge Mathematik
%%%%%%%%%%%%%%%%%%%
\usepackage{mathtools} % erweiterte Fassung von amsmath
\usepackage{amssymb}   % erweiterter Zeichensatz
\usepackage{siunitx}   % Einheiten

%%%%%%%%%%%%%%%%%%%
%Eigene Importe
%%%%%%%%%%%%%%%%%%%
\usepackage{amssymb}% http://ctan.org/pkg/amssymb
\usepackage{pifont}% http://ctan.org/pkg/pifont
\newcommand{\cmark}{\ding{51}}%
\newcommand{\xmark}{\ding{55}}%
\usepackage{svg}

%Formatierungen für Beispiele in diesem Dokument. Im Allgemeinen nicht notwendig!
\let\file\texttt
\let\code\texttt

\usepackage{pifont}% Zapf-Dingbats Symbole
\newcommand*{\FeatureTrue}{\ding{52}}
\newcommand*{\FeatureFalse}{\ding{56}}
%\bibliographystyle{IEEEtran}

\newcommand*{\mrm}[1]{\ensuremath{\mathrm{#1}}}

\bibliography{bib/Literatur}
%Own Commands



\linespread{1.25}





\begin{document}

\Metadata{
	title=Modeling and Analysis of Human Behavior Impacts on Energy Systems during Crisis Events,
	author=Isabella Nunes Grieser
}

\title{Modeling and Analysis of Human Behavior Impacts on 
Energy Systems during Crisis Events}
\subtitle{Modellierung und Analyse der Auswirkung des menschlichen
Verhaltens auf Energienetze während Krisensituationen}
\author[short Name]{Isabella Nunes Grieser}%optionales Argument ist die Signatur, 
%\birthplace{Geburtsort}%Geburtsort, bei Dissertationen zwingend notwendig
\reviewer{Prof. Dr. Florian Steinke \and M.Sc. Tobias Gebhard}%Gutachter
\publishers{}
%Diese Felder erden untereinander auf der Titelseite platziert. 
%\department ist eine notwendige Angabe, siehe auch dem Abschnitt `Abweichung von den Vorgaben für die Titelseite'
\department{etit} % Das Kürzel wird automatisch ersetzt und als Studienfach gewählt, siehe Liste der Kürzel im Dokument.
\institute{\includegraphics*[width=4cm]{gfx/EINS_Logo}}
\group{}

\submissiondate{\today}
\examdate{\today}

%	\tuprints{urn=1234,printid=12345}
%	\dedication{Für alle, die \TeX{} nutzen.}

\maketitle
\affidavit

\begin{abstract}[english]

In the COVID-19 pandemic, multiple instances where people rushed 
to buy certain goods excessively showed how vulnerable supply 
chains are to drastic changes in consumer demand.
These instances primarily involved physical goods such as toilet papers.
Nonetheless, critical infrastructure that deals with non-physical 
goods such as electricity or communication may also be victims of 
sudden spikes in consumer demand. Critical infrastructure
needs to be resilient towards various kinds of threats, it 
is important to analyse possible threats to the infrastructure
before they happen to create countermeasures.
Thus the effects of sudden demand changes should be analysed
in the context of energy systems. 




\end{abstract}
\begin{abstract}[ngerman]

\end{abstract}

\tableofcontents


%\listoftables
%\listoffigures
\printglossary %[type = \acronymtype]

\chapter{Introduction}

This Chapter aims to show the significance of information on social media and 
the changes in human behavior based on it
in the context of the electrical grid.
First, in Section \ref{contextmotivation}, the necessary context and 
motivation for this thesis is provided.
Then, the problem that is considered in this thesis is described 
in Section \ref{problemstatement}. Afterward, the contribution of this
thesis is presented in Section \ref{contribution}, 
followed by the outline of this work in Section \ref{outline}.

\section{Context and Motivation}
\label{contextmotivation}

During the COVID-19 pandemic, severe changes in consumer behavior
lead to massive supply issues, thus showing how drastic supply markets 
react to changes in consumer demand.
Consequently, there has been a growing interest in researching 
how shifts in demand manifest during crises. 
One reason for changes in consumer demand is 
social media usage. Consumers can receive new information
and change their demand on certain goods based on the information received 
through social media \cite{naeem2021social}.
Current research mostly considers regular, physical 
goods for these demand changes.
However, the results shown in these 
works may not translate well to critical infrastructure,
such as water, energy or communications systems,
since they have characteristics that differ from physical goods.

The power grid is a critical infrastructure 
that is an important for our society.
There are yet no examples of drastic changes in consumer behavior
based on information of social media leading to supply issues or
even the breakdown of the power grid. Nonetheless, 
there are still possible scenarios where the demand for
electricity may increase due to information on social media.
Moreover, the consequences of such events would be critical for the
reliability of the power grid.
For example, a false notice of reduced prices 
could lead to an unexpected demand increase which 
can be critical at times of peak demand. 
The additional demand
may increase the total demand to a level that the power grid
cannot support, thus leading to blackouts.
Another example would be that coordinated actions to reduce
power, either in regards to combat climate change \cite{earthday}
or as an attack against the state by conspiracy theorists,
could lead to a sudden, non-negligible mismatch of supply and demand.


\section{Problem Statement}
\label{problemstatement}

To reduce the probability of blackouts, 
the resilience of the power grid is of vital importance.
Resilience describes the ability of a system 
to maintain critical services, to recover from threats
and to adapt the system to future hazards 
\cite{wells2022modeling}.
To assure the resilience of the power grid, a variety of possible 
scenarios which could negatively affect the power grid need
to be analyzed. Moreover, countermeasures need to be made for all possible scenarios.
Thus, it is reasonable to analyze theoretical scenarios that, even though
they never happened until the present time, could plausibly happen. This 
would allow policymakers to draw countermeasures to these possible scenarios
and allow for quick and decisive actions to be taken by relevant 
parties.

Until now, there are already works which analyze the possible effects
of information, specially misinformation, on the power
grid. However, these works were of theoretical nature and
did not consider a detection system to warn of a possible 
failure of the power grid. Furthermore, these works did mainly analyze the 
possibility of a false notice of reduced electricity prices.
Other scenarios were usually not considered.

\section{Contribution}
\label{contribution}
This thesis makes multiple contributions to addressing the issue of 
power demand fluctuations due to social media.
First, this thesis both introduces and 
analyzes multiple different scenarios, in which
information on social media leads to a change in power demand.
Second, this thesis proposes a framework to simulate and predict if 
a certain topic is propagating through social media
in a speed that could lead to issues for the power grid.

\section{Outline}
\label{outline}

First, the necessary background for this thesis is explained
in Chapter \ref{background}. 
In Chapter \ref{relatedworks}, works related to 
this thesis are introduced.
Next, in Chapter \ref{implementationall}, the framework to
analyze and predict possible power demand surges based on
information on social media is introduced. Furthermore,
an algorithm to estimate parameters to realistically simulate
the effects of information on social media is introduced and 
the all steps of the framework are explained.
In Chapter \ref{results}, the different scenarios 
considered in this thesis are introduced and analyzed. Both the 
framework implemented in Chapter \ref{implementationall}
and the general risk of such scenarios are 
examined.
Last, in Chapter \ref{conclusion}, the conclusion drawn 
from this thesis is described. Moreover, the possibilities
of future research associated with this thesis 
are mentioned.
\chapter{Background}
\label{background}

In this chapter, the fundamental concepts necessary for this thesis are explained.

\section{The Electrical Power Grid as a Critical Infrastructure}

\subsection{Power Load as a Power Outage Risk}

\subsection{Social Network-based Attacks on the Electrical Power Grid}

\section{Graph as a Representation of Real-World Systems}
\label{graphbasics}
A graph is an abstract structure that represents a set of objects and the relationship 
between the objects. Graphs can be used to model a variety of real-world systems.
Notable examples of systems that can be modelled as graphs that are of 
interest in this thesis are social networks 
\cite{socialgraphexample} and power grids \cite{powergraphexample}, but graphs can 
also be applied to other real life systems, such as economic 
or transportation systems \cite{economicsgraph}. %TODO: cite for transportation


Per definition, a graph $G(V, E)$ consists of a set of vertices $V$ and a 
set of edges $E$. Each edge connects two vertices. This means that for all $E$, the 
condition  $E \subseteq\{ \{x, y\} \mid x, y \in V  \}$ 
is valid.

There are multiple different subtypes of graphs. A directional graph is a graph 
whose edges have a source node $a$ and target node $b$, thus representing
a direction going from $a$ to $b$. If the edges do not fulfill this 
condition, thus if the edges do not show a direction, then a graph is unidirectional.
In addition, a graph may have attributes, so-called weights, associated to its edges.
These graphs are called weighted graphs.

There are also multiple attributes that can be calculated in a graph. 
Relevant attributes are:
\begin{itemize}
    \item The degree $deg(v)$ of a vertex $v \in V$ is the number of edges connected to the
    vertex $v$. 
    \item The average path length is the distance between two vertices is
    defined as the number of edges along the shortest path
    connecting them %TODO: verbessern
    \item Clustering coefficient: The clustering coefficient, , CCi
    of a vertex i
     is the ratio between the actual number of edges
    that exist between the vertex and its neighbors and the
    maximum number of possible edges between these
    neighbors. The
    CC
    of the network is defined %TODO: verbessern
\end{itemize}

\subsection{Modelling Social Networks as Graph Systems}


\subsection{Random Graphs}

Graphs do not need to be based on real systems. They can also be generated. 
These randomly generated graphs are called random graphs \cite{randomgraphs}.



\section{Misinformation in the Social Media Age}

In recent years, there has been widespread concern that 
misinformation on social media is creating real-life damage in 
many different sectors of society and reducing the trust in govermental
institutions. Thus, the topic of online misinformation has become an 
important research field in the academic community.

\subsection{Information Diffusion in Social Networks}

One method to research the consequences of misinformation over social media
is to understand how the information spreads over
social networks and analyse occuring patterns in the propagation process. 
The spread of information over a given network is called
information diffusion. 
The modelling of information diffusion can be divided into three components
\cite{reviewinformationdiffusion}: 

\begin{itemize}
    \item State: Entities in a system may be grouped together in a class which
    defines their behavior in the system. In the context of information diffusion,
    an entity can for example belong to the  \glqq Spreader \grqq{} class which spreads the information
    to specific other entities in the system. An entity could also belong to a
    \glqq Susceptible \grqq{} class that did not receive the information, 
    but would possibly be able to receive it.
    \item Model: The model defines how the entities interact which each other
    given their states. A model is often represented by a graph with the entities
    as nodes and connections between the entities as edges 
    (See section \ref{graphbasics}).
    \item Transition rates: The transition rates are the parameters used
    do define how the information is diffused in the model dynamically. 
\end{itemize}

The first two components are independent of specific simulation scenarios,
unlike the third component, thus the first two components should be further
analysed. Furthermore, there are a multitude of different ways 
to model information diffusion in social networks. Therefore, the most common
modelling methods will be further analysed in this chapter.

\subsubsection{Epidemiological Models}

Due to their similarity to the spread of infectious diseases, 
the spread of misinformation is often modelled the same as epidemiological models.
The simplest epidemiological model is the SI model, where entities in a system
can either belong to the state \glqq Susceptible\grqq{} ($S$) or 
\glqq Infected\grqq{} ($I$) and where the amount of total entities of a 
system can be calculated as $N=S+I$. The SI model is the basis for many
more specialised models where the states are more detailed to capture more 
aspects of real life behavior. A few models based on the SI model and their 
specific state are shown in Table \ref{SI-table}.

\begin{table}[ht!]
    \centering
    \begin{tabular}{|c | c |} 
     \hline
     Acronym & Classes  \\ 
     \hline
     SIR & Susceptible - Infected - Recovered  \\ 
     \hline
     SIS & Susceptible - Infected - Susceptible \\
     \hline
     SEI & Susceptible - Exposed - Infected \\
     \hline
     SIRS & Susceptible - Infected - Recovered - Susceptible \\
     \hline
     SHIR & Susceptible - Hesitant - Infected - Recovered \\
     \hline
     SCIR & Susceptible - Contacted - Infected - Recovered \\
     \hline
    \end{tabular}
    \caption{Example models building on the SI 
    model for the spread of misinformation \cite{reviewinformationdiffusion}}
    \label{SI-table}
\end{table}

Of specific interest for this thesis are the SIS and SIR models. 
For the SIR model, it is assumed that at the beginning, all entities in the 
system are in the state $S$, thus susceptible to an infection. In the context of 
misinformation diffusion this means that an entity does not know about the
misinformation that is being spread and is neutral towards it. A susceptible
entity may become infected and thus may change its state $S\to I$.
An infected entity knows and believes in the misinformation and it may
spread the misinformation to other entities in the system. After being infected,
an entity may learn that the information is fake. Therefore, it recovers 
from being infected and it changes its state $I\to R$. An entity in a 
recovered state cannot be infected again by misinformation and is thus immune.
The SIS model differs from the SIR model in that in the last state change,
the entity does not become immune, but \glqq forgets\grqq{} about the fact
that the information is fake, and becomes susceptible to misinformation again.
%TODO: finde Grundlagenbuch für states und modellchanges

The amount of entities belonging to each state may change 
after a time duration $\Delta t$. The equations to calculate the 
state changes, and thus the progress of the information diffusion, 
can be seen in Table \ref{SI-table-equations}.

\begin{table}[ht!]
    \centering
    \begin{tabular}{|c | c |} 
     \hline
     SIR & SIS  \\ 
     \hline
     & \\
     $\begin{aligned}
          \frac{dS}{dt} &= -\beta I S \\
          \frac{dI}{dt} &= \beta I S - \gamma I \\
          \frac{dR}{dt} &= \gamma I  
        \end{aligned}$
      &
      $\begin{aligned}
          \frac{dS}{dt} &= -\beta I S + \gamma I\\
          \frac{dI}{dt} &= \beta I S - \gamma I
        \end{aligned}$
       \\ 
       & \\
     \hline
    \end{tabular}
    \caption{Model equations for SIR and SIS model \cite{sirequation}}
    \label{SI-table-equations}
\end{table}

$\beta$ and $\gamma$ are the transition rates parameters of the model, where 
$\beta$ is the transmission rate and $\gamma$ the 
recovery rate. Moreover, the equations in Table \ref{SI-table-equations}
are ordinary differential equation that are used to calculate the amount
of entities for each state in the system, not to calculate in which state
each entity in the system is in. To analyse the information diffusion on 
an entity level, graph based diffusion models are needed.
There is no exact definition for the algorithm for the state changes in the
graph model, but generally, the probability, that an entity changes
its state from susceptible to infected $S \to I$, increases depending on
how many connected entities are already infected. In its simplest form, a
propagation algorithm may be defined as in Equation \ref{eqbasicpropagation} 
\cite{easypropagation}.

\begin{equation}
    I(a) = 1 (\sum\limits_{\substack{(b,a)\in E, \\ b \in V \cap I}}
    1(f^{rand}\geq \beta)>0) 
    \label{eqbasicpropagation}
\end{equation}

In the equation, the infection status $I(a)$ of each node $a$
in the graph is defined by if any neighboring node $b$ that is connected 
to $a$ and is part of the infection node set $I$, 
and thus infected, fullfills the condition $f^{rand}\geq \beta$,
where $f^{rand}$ is the random probability generation function and $1$ 
is a function that equals one if the specific condition is fulfilled and 
zero otherwise.

%TODO: more stuff to write

\subsubsection{Information Cascade Models}

Another way to model information diffusion in graphs is by viewing it as a 
sequential information propagation process. This assumption is made in
information cascade models. In this type of model, any node $n$ infected in the
iteration step $i$ may infect its connected neighbor $a$ with a probability $P_n(a)$
\cite{reviewinformationdiffusion}. All nodes infected by node $n$
can then infect their neighboring node in the next iteration step $i+1$
and the node $n$ becomes inactive and does not infect any neighbors anymore.
The equation for the propagation step can be seen in Equation 
\ref{eqbasicpropagationcascading}.
Information cascade models are mostly used for prediction and influence 
analysis purposes, and not to explain the collective behavior
of the system during the information diffusion process.

\begin{equation}
    I(a) = 1 (\sum\limits_{\substack{(b,a)\in E, \\ b \in V \cap I}}
    1(f^{rand}\geq P_n(a))>0) 
    \label{eqbasicpropagationcascading}
\end{equation}

There are several different subtypes of cascading models.
Some subtypes are described in \cite{diffusionbasics}:

\begin{itemize}
    \item Independent Cascading Model: In this type of model, the 
    probability $P_n(a)=p$ ist constant for each node $a$ and neighbor $n$.
    Thus, the probability function does not depend on the history 
    of the system and its information diffusion status.
    \item Decreasing Cascading Model: In this type of model, the probability
    function to activate the node $a$ decreases with each attempts of its 
    neighbors. This means that if a neighbor $n$ unsuccessfully tries to infect
    $a$ at iteration step $i$, then the probability that the neighbor $m$
    can sucessfully infect $a$ at step $i+t$ is smaller, thus $P_n(a)>P_m(a)$.
\end{itemize}

\subsubsection{Threshold Models}
A different method to analyse the diffusion process in graphs is by seeing the
propagation step as a process where each entity needs to overcome a 
specific threshold $\theta$ to become infected. More specifically, 
a certain amount of neighbors need to be infected for the node $a$ to become 
infected, too. Furthermore, contrary to the cascade models, a node can always 
infect its neighbors as long as the conditions are fulfilled in the threshold 
model. The general equation for the threshold model can be seen in Equation
\ref{eq:threshold}.

\begin{equation}
    I(a) = \sum\limits_{\substack{(b,a)\in E, \\ b \in V \cap I}}
    1 > \theta    
    \label{eq:threshold}
\end{equation}

Subtypes of this kind of diffusion model differ by the threshold parameter.
Some subtypes are described in \cite{diffusionbasics}:

\begin{itemize}
    \item Majority Threshold Model: In this type of model, a node $a$ becomes
    infected if the majority of its neighbors are infected, thus 
    $\theta = \frac{1}{2}deg(a)$
    \item Small Threshold Model: In this type of model, the threshold for that
    $a$ becomes infected is small. The advantage of this model is that 
    certain algorithms may be calculated faster with this type of model 
    \cite{diffusionbasics} 
    \item Unanimous Threshold Model: In this type of model, all neighbors 
    of a node $a$ need to be infected for $a$ to become infected, thus
    $\theta = deg(a)$.
\end{itemize}

\section{Variable Dependencies in the Context of Power Usage}

The power demand is always changing. The changes in demand depend on a multitude 
of factors and variables and can change drastically over the day and over the year.
These variables can be classified as either endogenous or exogenous variables.
Exogenous variables can be considered as independent variables 
whose values determined outside of the system. 
Endogenous variables, on the other side, 
are variables that are dependent of other variables in the 
system.

In the context of power load prediction, the most commonly 
used endogenous variable is the historical power load data.
For the exogenous variables, there are multiple types of 
variables that are often considered as possible input data \cite{exogenousdata}
\cite{exogenousdata2}.
The first type of variables are environmental variables like temperature, 
rainfall, humidity or wind power. A different type of variables are time data
such as the weekday, if a day is a holiday and the time of the day.
One more type of variables are socio-economic variables such as the
population size and growth, the exchange rate, the income level,
the gross domestic product or the different types and amount 
of consumers like agricultural, industrial or household consumers.
Another type of variables are building and occupancy related variables such
as the household appliance usage,
the number of persons or the number of bedrooms in a household.

The relevance of each type of variables depend heavily on it is
a short-, middle- or long-term power load forecasting. When shifting to more 
long-term predictions, the slower changing variables such as socio-economic
variables become more important than short-term variables such as weather data 
\cite{loadforecastingtimedependency2}\cite{loadforecastingtimedependency}.
The specific task definition is also relevant for the parameter selection.
Power load prediction models for residential buildings may benefit from 
building and occupancy related variables, but for a prediction model on a 
macroeconomic scale, socio-economic variables are more important.

\subsection{Exogenous Variables based on Human Behaviour}

Exogenous variables that are explicitly linked to human behaviour, 
such as social media usage ,traffic information, 
% ICT location, IoT information (such if a device is turned on or off), 
satellite image data or internet usage, are not often 
used for power load forecasting. 
But these are the variables that would be affected the most in a change of 
human behavior. Thus, these variables should be analysed in more detail 
for this thesis.

Social media data does not directly correlate with power demand, but it is
possible to extract information relevant for power demand prediction
from the content generated by it:
First, it is possible to analyse the spacial density of people by 
counting the amount of tweets tagged in a specific location of interest. 
In the work of \textit{Deng et al.},
it was shown that there is a significant
correlation between the amount of geotagged tweets
and the power consumption in a specific region \cite{twittergeoloccorr}.
As a possible explanation for this correlation, the paper assumes that 
an increased amount of human activity in a specific region leads a greater 
use of facilities such as heating or air conditioning in buildings or 
other types of power-consuming behavior.
For this reason, geotagged tweets were already used as input data for 
power load forecasting models 
\cite{twittergeolocforecasting} \cite{twittergeolocforecasting2}.
Second, it is also possible to analyse the content written in the 
social media posts. The content in social media posts can be 
analysed in two ways: 
It can either be analysed by searching for posts with a defined keyword or tag
or it can be further analysed by using natural language processing techniques. 
In natural language processing the raw text content 
(and possibly its metadata and attachements) of a post is analysed
in some form to extract the desired information. Possible analysis
methods would be to classify either a post is related to some energy-consuming
behavior or to count the number of energy-relevant keywords in the text.
For example, a natural language processing model was trained to 
classify specifically energy-related tweets \cite{energybert}.
The other possibility is to query the social media instance 
to find all posts that contain the defined keyword or tag.
This method is usually simpler than using more sophisticated natural 
language processing techniques, but increases the amount of false positive 
posts that are not relevant for the specific task.
For power load prediction tasks, only keyword search is used.
In the work of \textit{LaLone et al.}, they show that the amount of energy
related tweets during hurricane Katrina correlated with the amount of 
power outages at the same time \cite{poweroutagetwitter}.
Another paper uses twitter data with both keyword search and 
tweets that were classified with natural language processing techniques to
to find power outage regions \cite{twitterpoweroutagelighttime}.
In addition, another paper used topic mining techniques to find
events on twitter posts that correlate with energy consumption 
\cite{twittertopicevent}.
A data type that can be considered as partially related to social 
network data, since it also shows the interest that greater parts of 
the population have for a specific topic, are search engine indices.
The work of \textit{Wu et al.} shows that adding keyword search counts
can improve power load forecasting accuracy \cite{googlepowerforecast}.

General internet traffic data also correlates with power usage.
In the study of \textit{Morleya et al.}, they conclude that 
electricity consumed by information and communication devices
may contribute to increased peak electricity demand 
\cite{internettrafficenergycorrelation}.
%TODO: describe how much ICT uses
%TODO: find corellation between internet traffic and people 
%staying at home
Therefore, internet traffic data can be used as input data for 
power demand forecasting models \cite{electricityinternetforecast}. 

Studies show that with the increased adoption of electric vehicles (EV), 
there is an increased interdependency between the electric and 
transportation infrastructure \cite{interdependnytrafficenergy}. 
Therefore, traffic data may correlate with power load 
in two ways:
First, with the traffic movement in general, it is possible to follow the
movement patterns of larger group. As an example, in the morning,
traffic intensity increases since larger part of the population needs
to go to their workplace. Subsequently, this population group
returns back to their residences after finishing their work in the afternoon.
Thus, there is also an increased traffic intensity in the afternoon.
This traffic movement can be analysed by measuring the 
incoming and outgoing traffic volume at the main roads of the 
region of interest. The usage of this variable in power load
forecasting models gained more interest during the COVID-19 pandemic, 
where drastically changing mobility and power consumption 
patterns meant that new model inputs were necessary to consider changing
behavior patterns in the population 
\cite{covidtrafficpower} \cite{covidtrafficpower2}.
But this variable also correlates with the electricity consumption pattern
outside of pandemic periods. One paper successfully showed that
using traffic data as an input for power load
forecasting models generally leads to more accurate predictions, even before
the COVID-19 pandemic \cite{causalmodeltrafficelectricity}.
Second, with the increasing EV usage in the future, 
the total power demand increases since these types of vehicles 
need to be charged often. Power load forecasting methods that consider this
issue often focus on charging stations \cite{evcharchingstations}
\cite{evcharchingstations2}, but changes in household power consumption 
and the general charging behavior of households
were also the objective of some studies. \textit{Gerossier et al.}
analysed in their study the different charging patterns in households
and found four different charging patters in which its members charged at 
a similar time and duration \cite{gerossier2019modeling}.
In addition, there are also studies that work on
power load prediction models for individual household
charging forecasts \cite{skala2023interval}. 
A notable study was done by \textit{Arias et al.} that also considered the traffic patterns
and analysed the differences in EV charging behaviors between
commercial and residential districts when forecasting the power demand
created by EV charging \cite{arias2016electric}.

The last variable that depends on human behavior mentioned in this section are
satellite images. These kind of images can be used to find a variety 
of enviromental factors that are of relevance for the amount of power demand.
Satellite images are mostly used to forecast power generation of 
renewable resources such as solar power \cite{solarprediction}.
But it can also be used to estimate household power consumption,
specially during nighttime. This is specially of relevance for regions 
where the electricity data is unreliable, such as in developing contries
\cite{reviewnighttime}. There is a relation 
between nighttime light intensity and power consumption.
But this relation depends on the type of consumer 
(households, industry, ...). Furthermore, the relation is not linear
\cite{nighttimepowerestimation}. 
A disadvantage of using nighttime images for power demand prediction is 
the inherent noise in the data. Non-household light sources such as 
streetlights and vehicles add noise to the image data \cite{reviewnighttime}.
Another usage of nighttime satellite image is for power outage detection.
If there is a power outage, the affected region will turn dark, which 
consequently can be detected in satellite images. Thus, sattelite image data 
was successfully used in multiple power outage detection models
\cite{nightpoweroutage} \cite{twitterpoweroutagelighttime}.

\chapter{Related Works}

In light of the increasing usage of web-based technologies in our daily 
lifes, web-based points of attacks that could attack a country's 
electrical infrastructure and other critical infrastructure 
are becoming an increasing concern.
Thus, the resilience of the electrical grid is in danger.

%One of these points of attack created by web-based technologies
%are direct attacks via cyber attacks. The increased usage of 
%smart grids, which uses the internet to support a variety of functions 
%to make the electrical grid more efficient, increases the 
%vulnerability of the electrical grid to cyberattacks.
%Thus, cyber security for smart grids is a widely researched subject.
%The threat of cyberattacks are research from multiple angles.
%First, general vulnerabilities of the smart grid and the threats 
%created by possible cyberattacks can be analysed and discussed 
%\cite{cyberbasic1} \cite{cyberbasci2}.
%Second, the detection of cyber attacks on the electrical grids are 
%a wide resarch field. 

%Hier könnte man auch false pricing attack power grid als keyword noch
%weitersuchen
The point of attack relevant for this thesis are social network-based
attacks. These attacks try to influence the behavior of groups of people to 
change the electricity consumption on a scale that may threaten the reliability
of the electrical grid. This type of attack is not as well researched 
compared to cybersecurity-based attack points. All works in this topic
mainly consider false pricing attacks, where consumers receive false 
electricity pricing information over social media and change their 
electricity demand pattern based of the false information.
In the work of \textit{Tang et al.}, they model such kind of attack on the 
smart grid with a complex information propagation algorithm which considers 
multiple types of influences, consumers with different types of 
personalities and other characteristics \cite{falsepricing1}.
Furthermore, they analyse the possible reaction to the attack, where
the operators use load shedding to reduce the load on the energy system. 
Another work implements a model similar to the model
proposed by \textit{Tang et al.}, but with a focus on the consequences of a 
possible false pricing attack \cite{vulnerabilityanalysis}.

Another problem that can be analysed in this topic is which $k$ entities 
need to be influenced by misinformation to instill 
the maximum damage on the power grid. In the work of \textit{Pan et al.}, 
they analysed this issue, developing several heuristics to find the most 
relevant entities in the system that can inflict the maximum damage by both 
propagating the misinformation and increasing the demand 
$d$ to $d(1+\Delta i)$ \cite{pan2017threat}. 
Furthermore, they analysed load shedding as a
countermeasure for the targeted misinformation attack.
\textit{Nyugen et al.} extended the work of \textit{Pan et al.}
by further analysing the general vulnerability and also the resilience of 
the power grid if automatic load shedding is applied to the overloaded grid
\cite{nguyen2019vulnerability}.

A paper that tries to analyse a more practical development of a misinformation
attack is the paper of \textit{Raman et al.} \cite{raman2020weaponizing}.
In their work, they used a survey to analyse the probability of a person
reacting to misinformation, and thus increasing the electricity demand,
and how likely they are to forward the misinformation to their peers.
Furthermore, they focused on possible impacts on the electrical grid in 
the future by also analysing the consequences of the increased 
usage of electric vehicles in the future. Possible countermeasures like
load shedding were not analysed.

Last, in the work of \textit{Jamalzadeh et al.}, they created a model to monitor
the power grid when its under a misinformation attack and to analyse
how a possible campaign to counter the misinformation attack can mitigate
the impacts of the attack \cite{jamalzadeh2022protecting}. Furthermore,
they describe an optimization algorithm to minimize the amount of 
entities affected by a possible blackout caused by the attack as 
a countermeasure mechanism.
\chapter{Implementation of a Framework to model the Effect
of Information on Social Media on the Power Grid}

In this chapter, a framework to analyse the effects of information on the 
power grid will be introduced and the specific implementation of the 
framework will be explained.
First, in Section \ref{simulationframeworksection}, 
the implementation of the simulation framework will described.
For this, the model for the social media network structure,
will be introduced in Section \ref{socialmediaimplementation}.
Then the information diffusion model, used to model the information 
propagation process in the system, will be explained in Section 
\ref{infodiffusionimplementation}. Afterwards, the method to 
model the excess power consumption will be introduced in Section 
\ref{powerconsumptionimplementation}. 

\section{Implementation of the Simulation Framework}
\label{simulationframeworksection}
There are multiple components that are essential to model the effects of 
both true and false information on critical infrastructures
such as the power grid. First, it is necessary to model an 
exemplary 
\subsection{Modelling of the Social Media Network Structure}
\label{socialmediaimplementation}



\subsection{Modelling of the Information Diffusion Process}
\label{infodiffusionimplementation}

\subsection{Modelling of the Excess Power Consumption}
\label{powerconsumptionimplementation}


\section{Extending the Framework to estimate the System Parameters based on
Domain Variables}


\subsection{Usage of analogous Social Media Data to simulate the Effects of 
Misinformation on the Power Grid}
\chapter{Case Studies}

%do the usual introduction
% then explain wtf the idea is
\section{Scenario 1: Demand-Response Misinformation Campaign}
\label{demandresponsesection}
%First: Explain throuhout the scenario
%Then define assumptions
%Then define parameters
%Then simulation results
%This Scenario should also be the one with the total framework

With the increased usage of information and communication 
technologies (ICT), new methods of influencing
consumer demand patterns are possible. 
These can be used by electrical companies to change the 
demand curve over time, ideally to reduce peak loads 
and thus increasing the efficiency of the electrical grid.
A realistic power demand, as shown in Figure \ref{duckcurve}, tend 
to vary greatly during the day. This leads to inefficiency,
since excess supply sources need to be build to satisfy customers
during peak hours. The idea is to lead consumers to shift
their consumption from peak demand hours to low demand hours,
thus reducing the variation of demand over time and allowing
for more efficient management of the power grid infrastructure.
This concept of changing the demand to match the 
available supply of electricity is called demand-response. 
One method to influence the consumer demand is by real time 
pricing (RTP). With RTP, the electricity prices change
dynamically based on the total electricity demand.
The price changes are then forwarded to the customers with ICTs, 
who are thus informed of the price changes.
The idea of RTP is that consumers change their electricity demand
by shifting activities with a high electricity usage, 
such as washing clothes or cooking, to time periods 
where the total electricity demand and thus the prices
are lower. This leads to reduced peak energy consumption. 
RTP was already used in a variety of programs
under real life conditions \cite{barbose2004survey}.
Studies showed the benefits of such programs \cite{albadi2008summary}.

\begin{figure}[!ht]
    \center
    \includegraphics[scale=.75]{figs/duckcurve.png}
    \caption{real power load and the ideal power load for the 
    power grid infrastructure}
    \label{duckcurve}
\end{figure}

Problematic is if a false pricing rumor spreads through social media.
In 2022, changes in the verification policies of Twitter lead to users being able 
to impersonate well-known brands such as Pepsi, leading to Twitter 
users being unable to determine which social media channels were
posting trustworthy information \cite{twitterchaos}. One user could
have used the policy changes of Twitter to impersonate 	utility companies
and to advertise a false reduction of energy prices, leading to customers
believing the information and increasing their energy consumption as a 
reaction. Another possibility would be that hackers would force access to the
social media profiles of 	utility companies and start sending false 
information to its customers. Hackers were already able to gain access
to the social media profiles of multinational companies in the past
\cite{twitterhacker}. Hackers could also spread misinformation in regards
to a false reduction in prices, thus leading to changing consumer demand.

% HERE AN EXAMPLE TWITTER POST

Based on the information, a possible scenario could be that hackers would
access the social media profiles of a well-known utility company
to spread the information that the energy prices would be reduced by 
a signifiant factor at a specific time. This information would then be 
spread around social media, reaching many customers of the company.


\section{Scenario 2: Coordinated Action of Extremists 
against the Electrical Grid}

Echo chambers on various social media websites allow people to 
hear biased opinions and news which confirm their views on 
various topics \cite{terren2021echo}. This leads to political
polarization and extremism \cite{van2022banality}.
These people in turn can try to fight the established political
system. Research shows that social media influence people
to commit hate crimes \cite{muller2021fanning}.
In general, the number of violent incidents caused by
extremists, specially right-wing extremists, is rising 
\cite{koehler2016right}. 
Most of the criminal acts caused by political extremism 
in germany are in the fields of property damage, 
propaganda crimes, insults and hate speech \cite{bmicrimestatistics}.
But extremists also cause more serious crimes. In 2022,
right-wing extremists planned to sabotage the electrical grid 
infrastructure by destroying power lines \cite{anschlagstrom}.
Thus, it can be seen that critical infrastructure may also be a target
for political extremists. 

Another way to target critical infrastructure is with a coordinated
actions which in its sum can damage the infrastructure. The 
reasons for a group of people to take such action may be harmless,
such as the Earth day, where people where encouraged to switch off their 
lights and appliances to save energy \cite{earthday}.
But extremists could also synchronize their actions to reduce their
power consumption to a minimum, thus mismatching electricity supply
and demand by a considerable degree. Electricity companies would 
then need to deal with a sudden surplus of electricity.

Given the previous information, a possible scenario could be 
that extremists plan to turn off all their electrical devices at 
a specific time to destabilize the electrical infrastructure. The
time would be announced in advance.


\section{Scenario 3: Mass Evacuation of a City due to a Disaster}

Sometimes the conditions under a disaster get so extreme that 
authorities choose to order the evacuation of the affected population.
For these types of evacuation, the autorities may use emergency channels 
to inform the population of the evacuation.
But people may also choose to leave the city in masses on their own accord.
The information of a possible disaster, such as a wildfire, reaching the
city soon or other disasters such as a widespread terrorist attacks 
in the city.
If people decide to leave their homes to flee from the disaster, they 
will mostly use their cars as the mode of transportation.
In the future, the increased adoption of electric vehicles (EV)
may lead to people needing to recharge their EVs if they wish 
to drive them. An evacuation could lead to many people charging
their cars at the same time, thus creating excess demand that 
the infrastructure is unable to handle. 

Given the previous assumptions, a possible scenario could be that 
rumors about a wildfire spreading towards a city leads to people
trying to leave the city in masses. In addition, a significant 
amount of people drive EVs, thus giving them the need to charge
their cars before they are able to leave the city.


\section{Scenario 4: Mass Showering after a Chemical Accident}
%This Scenario should also be the one with the total framework

With the increasing usage of ICTs in the population, it allows for 
new methods to communicate with people about extreme circumstances
such as natural disasters or other catastrophes. In 2022,
a new warning system called Cell Broadcast was introduced in Germany
\cite{techrichtlinie}. Cell Broadcast can be used to warn the population
of emergencies by sending all cellphone users in the 
region of interest a message with a loud alert tone.

Germany has one of the biggest chemical industry sectors in the world.
It is home to big chemical factories in cities such as Ludwigshafen and
Darmstadt. Areas with chemical factories have the risk of being
victims of chemical accidents.

A possible scenario would be that a chemical accident in a 
factory near a major city would lead to dangerous chemicals 
leaking out. These chemicals would spread through the air 
and adhere to the skin of humans, leading to serious 
health problems if they stayed on the skin for too long.
Thus, the city's government would use Cell Broadcast to
instruct the population to close their windows and 
to take a shower to remove the chemicals from their bodies.
Therefore, most people would take showers at the same time.


% Questions: how to define the parameters
% For some, I can use a dataset and use this as an analogy
% But e.g. scenario 2 it doesnt work well
% how can I put thing out of my ass
% what could be interesting parameter tweakings 
\chapter{Outlook}
A short summary of the thesis follows an contains what was the problem under focus, how was it tackled and what was achieved in the thesis. At last, further suggestions for future research shows the comprehension of the own work as well as the scientific procedure of ever refining a theory till it becomes a model.


\section{Usage of analogous Social Media Data to simulate the Effects of 
Misinformation on the Power Grid}

%\appendix

%\input{inc/diaApx.tex}


		
\printbibliography
	
\end{document}
