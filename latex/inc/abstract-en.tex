\begin{abstract}[english]


Multiple examples have shown how vulnerable supply 
chains are to unexpected changes in consumer demand in recent history.
Information on social media may be the cause of changes in demand by 
influencing consumers to panic buy certain goods.
Until now, these events primarily involved physical goods.
However, critical infrastructure that deals with non-physical 
goods such as the power grid could also be affected by 
sudden spikes of consumer demand. Critical infrastructure
needs to be resilient towards various kinds of threats, hence it 
is important to analyze possible threats to the infrastructure
before they happen to prepare countermeasures.
In this thesis, the effects of demand changes based on information
which is being spread are modelled and analyzed with different scenarios. 
A simulation framework is implemented, with a graph-based social network model 
and a epidemological model based on the 
susceptible-infectious-recovered model (SIR) to 
model information propagation processes. 
Furthermore, information propagation parameters 
are estimated with social media data. 
Then, four theoretical scenarios are used as examples
to analyze sudden changes in demand for the power grid.
The analysis shows that simultaneous shifts 
in consumer demand could pose risks to the 
stability of the power grid. 
Additionally, further adoption of 
energy-intensive technologies 
such as electric vehicles 
are shown to amplify the impact of 
synchronized changes in consumer 
demand on the energy infrastructure.


\end{abstract}