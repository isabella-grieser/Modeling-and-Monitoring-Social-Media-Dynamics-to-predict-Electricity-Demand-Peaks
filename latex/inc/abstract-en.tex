\begin{abstract}[english]

In the COVID-19 pandemic, multiple instances where people rushed 
to buy certain goods excessively showed how vulnerable supply 
chains are to drastic changes in consumer demand.
These instances primarily involved physical goods such as toilet papers.
Nonetheless, critical infrastructure that deals with non-physical 
goods such as electricity or communication may also be victims of 
sudden spikes in consumer demand. Critical infrastructure
needs to be resilient towards various kinds of threats, it 
is important to analyse possible threats to the infrastructure
before they happen to create countermeasures.
Thus the effects of sudden demand changes should be analysed
in the context of energy systems. 




\end{abstract}