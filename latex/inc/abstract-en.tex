\begin{abstract}[english]

In the COVID-19 pandemic, multiple instances where people rushed 
to buy certain goods excessively showed how vulnerable supply 
chains are to drastic changes in consumer demand.
These instances primarily involved physical goods such as toilet papers.
Nonetheless, critical infrastructure that deals with non-physical 
goods such as electricity or communication may also be victims of 
sudden spikes in consumer demand. Critical infrastructure
needs to be resilient towards various kinds of threats, it 
is important to analyze possible threats to the infrastructure
before they happen to create countermeasures.
Thus the effects of sudden demand changes should be analyzed
in the context of energy systems. 
One way these sudden demand spikes happen is by rumors and information 
on the internet. Consumers receive information which leads to them 
abruptly changing their consumer behavior. In this Thesis, the 
effects of demand changes based on information on the internet are 
modelled and analyzed given different scenarios. The model
is based on a graph-based structure to model social media networks
and a epidemological model (based on the 
susceptible-infectious-recovered model, or SIR)) 
to model information propagation processes. 
Furthermore, certain model parameters are estimated with social 
media data. 
Then, four theoretical scenarios were used as examples
to analyze both the simulation framework and 
possible overconsumption events in the context
of the energy infrastructure. 
The analysis showed that simultaneous shifts 
in consumer demand could pose a risk the 
stability of the power grid. 
Additionally, the further adoption of 
energy-intensive technologies 
such as electric vehicles may 
amplify the impact of 
synchronized changes in consumer 
demand on the energy infrastructure.


\end{abstract}