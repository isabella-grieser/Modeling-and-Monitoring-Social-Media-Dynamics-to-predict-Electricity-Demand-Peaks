\chapter{Related Works}
\label{relatedworks}
In light of the increasing usage of web-based technologies in our daily 
lifes, web-based points of attacks that could endanger a country's 
electrical infrastructure and other critical infrastructure 
are becoming an increasing concern.
Therefore, in order to secure the resilience of the electrical grid,
it is necessary to analyze these points of attack.

This Thesis deals with social media-based
attacks. These attacks try to 
influence the electricity consumption of households 
on a scale that may threaten the reliability
of the electrical grid. 
This type of attack is not as well researched 
compared to cybersecurity-based attacks
\cite{sun2018cyber}. 
Most works dealing with social network-based
attacks mainly consider false pricing attacks, where consumers receive false 
electricity pricing information over social media and change their 
electricity demand pattern based of the false information.
Other scenarios are usually not considered.

Social media-based attacks are still theoretical scenarios.
Hence, the related works considered in this Thesis use theoretical
models to analyze the effects of information on social media 
on the power grid.

The focus of the analysis can differ depending on the work.
\cite{vulnerabilityanalysis} mainly focus in how the influence 
level of different social media users can influence the 
impact on the power grid. The work in \cite{pan2017threat}
searches for $k$ entities that
need to be influenced by misinformation to instill 
the maximum damage to the power grid.
Other works only analyze the 
effects of misinformation on the power grid
in general.
Moreover, some works analyze possible countermeasures to misinformation-based
attacks. Most of the countermeasures analyze load shedding as a 
countermeasure \cite{pan2017threat} \cite{nguyen2019vulnerability}.
In addition, most works do not propose any type of framework to 
monitor or predict changes in power demand based on 
information on social media.

The information propagation algorithm used in the model also differs based
on the work.
Most models use independent cascade algorithms for the propagation
\cite{raman2020weaponizing} \cite{pan2017threat}
\cite{nguyen2019vulnerability}. 
However, other algorithms were also used. One work uses an SIR 
model as an information propagation algorithm 
\cite{jamalzadeh2022protecting}. Amother work 
\cite{vulnerabilityanalysis} introduces a price-based multi-level 
influence propagation algorithm. In this algorithm, the information
propagation depends on how much the entities want to influence other
entities and the utility in the message being forwarded. 

A notable mention is the work in \cite{raman2020weaponizing},
where the theoretical scenario of misinformation influencing 
the power demand is analyzed with real data. For their work,
\textit{Raman et al.} used a survey to analyze the probability 
of a person reacting to misinformation,
and thus increasing the electricity demand.
Furthermore, they analyze how likely the affected entities 
are to forward the misinformation to their
peers. Other works usually do not use any real data to 
make their models more realistic.

Last, most works try to map the social media model to 
a phyiscal power grid. Hence they map virtual social 
media profiles to physical, energy-consuming 
households that are connected to the energy infrastructure.
The mapping is usually done by randomly mapping 
social media entities to physical entities described
in real power grid datasets \cite{nguyen2019vulnerability}
\cite{pan2017threat}. However, this may not show a realistic 
mapping between the two structures. There is 
research into the problem of mapping social media 
posts to physical locations \cite{jurgens2015geolocation}.
A geolocation prediction frameworks that 
uses Facebook household address data as a training data 
source is introduced in \cite{backstrom2010find}.
Twitter data can also be used to predict the location
of the residence of an user \cite{zheng2018survey}.
However, the accuracy of the detailed location of 
an user's home location is questionable.
