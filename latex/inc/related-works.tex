\chapter{Related Works}
\label{relatedworks}
In light of the increasing usage of web-based technologies in our daily 
lifes, web-based points of attacks that could endanger a country's 
electrical infrastructure and other critical infrastructure 
are becoming an increasing concern.
Therefore, in order to secure the resilience of the electrical grid,
it is necessary to analyze these points of attack.

This thesis deals with social media-based
attacks. These attacks try to 
influence the electricity consumption of households 
on a scale that may threaten the reliability
of the electrical grid. 
This type of attack is not well researched compared to
cybersecurity-based attacks
\cite{sun2018cyber}. 
Most works dealing with social network-based
attacks mainly consider false pricing attacks. In this type of attack, 
consumers receive false 
electricity pricing information over social media and change their 
electricity demand based of the false information.
Other scenarios are usually not considered.

The presumed risk of social media-based attacks is that a 
great amount of people
use an additional amount of energy synchronously.
Synchronized behavior can be described 
as the phenomenom where different individuals
consciously or unconsciously
perform the same actions at the same time.
This problem is already known and was
already the topic of other research works
\cite{lei2014impact} \cite{walker2014dynamics}
\cite{gebhard2022monitoring}.

Social media-based attacks are still theoretical scenarios.
Hence, the related works considered in this thesis use theoretical
models to analyze the effects of social media information
on the power grid.

The work in \cite{vulnerabilityanalysis} mainly focuses in how the influence 
level of different social media users can influence the 
impact on the power grid. Another work
searches for $k$ entities that
need to be influenced by misinformation to instill 
the maximum damage to the power grid \cite{pan2017threat}.
Other works only analyze the 
effects of misinformation on the power grid
in general.
Moreover, some works analyze possible countermeasures to misinformation-based
attacks. Most of these works analyze load shedding as a 
countermeasure \cite{pan2017threat} \cite{nguyen2019vulnerability}.
In addition, most works do not propose any type of framework to 
monitor or predict changes in power demand based on 
information on social media.

The information propagation algorithm used in the model may also differ.
Most models use independent cascade algorithms for the propagation
\cite{raman2020weaponizing} \cite{pan2017threat}
\cite{nguyen2019vulnerability}. 
However, other algorithms were also used. One work uses a SIR 
model as an information propagation algorithm 
\cite{jamalzadeh2022protecting}. Another work 
\cite{vulnerabilityanalysis} introduces a price-based multi-level 
influence propagation algorithm. In this algorithm, the information
propagation depends on how much the entities want to influence other
entities and the utility in the message being forwarded. 

A notable mention is the work of \cite{raman2020weaponizing},
in which the theoretical scenario of misinformation influencing 
the power demand is analyzed with real data. The authors 
used a survey to analyze the probability 
of a person reacting to misinformation.
Furthermore, they analyze the likelihood of the affected entities 
to forward the misinformation to their
peers. Other works usually do not use any real data to 
enhance their models.

Most works try to map the social media model to 
a phyiscal power grid. Hence, these works map virtual social 
media profiles to physical, energy-consuming 
households that are connected to the energy infrastructure.
The mapping is usually done by randomly connecting 
social media entities to physical entities described
in real power grid datasets \cite{nguyen2019vulnerability}
\cite{pan2017threat}. However, this may not constitute a realistic 
mapping between the two structures. 
The challenge of 
associating social media posts with specific geographical locations
is currently an active research field \cite{jurgens2015geolocation}.
A geolocation prediction model that 
uses Facebook household address data as a training data 
is introduced in \cite{backstrom2010find}.
Twitter data can also be used to predict the location
of the residence of users \cite{zheng2018survey}.
However, determining the detailed location of 
users' home location is difficult, specially with twitter data.
