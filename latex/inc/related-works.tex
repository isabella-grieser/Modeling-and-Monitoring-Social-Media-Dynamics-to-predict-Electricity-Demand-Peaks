\chapter{Related Works}

In light of the increasing usage of web-based technologies in our daily 
lifes, web-based points of attacks that could attack a country's 
electrical infrastructure and other critical infrastructure 
are becoming an increasing concern.
Thus, the resilience of the electrical grid is in danger.

%One of these points of attack created by web-based technologies
%are direct attacks via cyber attacks. The increased usage of 
%smart grids, which uses the internet to support a variety of functions 
%to make the electrical grid more efficient, increases the 
%vulnerability of the electrical grid to cyberattacks.
%Thus, cyber security for smart grids is a widely researched subject.
%The threat of cyberattacks are research from multiple angles.
%First, general vulnerabilities of the smart grid and the threats 
%created by possible cyberattacks can be analysed and discussed 
%\cite{cyberbasic1} \cite{cyberbasci2}.
%Second, the detection of cyber attacks on the electrical grids are 
%a wide resarch field. 

%Hier könnte man auch false pricing attack power grid als keyword noch
%weitersuchen
The point of attack relevant for this thesis are social network-based
attacks. These attacks try to influence the behavior of groups of people to 
change the electricity consumption on a scale that may threaten the reliability
of the electrical grid. This type of attack is not as well researched 
compared to cybersecurity-based attack points. All works in this topic
mainly consider false pricing attacks, where consumers receive false 
electricity pricing information over social media and change their 
electricity demand pattern based of the false information.
In the work of \textit{Tang et al.}, they model such kind of attack on the 
smart grid with a complex information propagation algorithm which considers 
multiple types of influences, consumers with different types of 
personalities and other characteristics \cite{falsepricing1}.
Furthermore, they analyse the possible reaction to the attack, where
the operators use load shedding to reduce the load on the energy system. 
Another work implements a model similar to the model
proposed by \textit{Tang et al.}, but with a focus on the consequences of a 
possible false pricing attack \cite{vulnerabilityanalysis}.

Another problem that can be analysed in this topic is which $k$ entities 
need to be influenced by misinformation to instill 
the maximum damage on the power grid. In the work of \textit{Pan et al.}, 
they analysed this issue, developing several heuristics to find the most 
relevant entities in the system that can inflict the maximum damage by both 
propagating the misinformation and increasing the demand 
$d$ to $d(1+\Delta i)$ \cite{pan2017threat}. 
Furthermore, they analysed load shedding as a
countermeasure for the targeted misinformation attack.
\textit{Nyugen et al.} extended the work of \textit{Pan et al.}
by further analysing the general vulnerability and also the resilience of 
the power grid if automatic load shedding is applied to the overloaded grid
\cite{nguyen2019vulnerability}.

A paper that tries to analyse a more practical development of a misinformation
attack is the paper of \textit{Raman et al.} \cite{raman2020weaponizing}.
In their work, they used a survey to analyse the probability of a person
reacting to misinformation, and thus increasing the electricity demand,
and how likely they are to forward the misinformation to their peers.
Furthermore, they focused on possible impacts on the electrical grid in 
the future by also analysing the consequences of the increased 
usage of electric vehicles in the future. Possible countermeasures like
load shedding were not analysed.

Last, in the work of \textit{Jamalzadeh et al.}, they created a model to monitor
the power grid when its under a misinformation attack and to analyse
how a possible campaign to counter the misinformation attack can mitigate
the impacts of the attack \cite{jamalzadeh2022protecting}. Furthermore,
they describe an optimization algorithm to minimize the amount of 
entities affected by a possible blackout caused by the attack as 
a countermeasure mechanism.