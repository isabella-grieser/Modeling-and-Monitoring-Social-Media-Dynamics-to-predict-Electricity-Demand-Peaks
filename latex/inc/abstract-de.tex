\begin{abstract}[ngerman]

Während der COVID-19-Pandemie gab es 
mehrere Fälle, bei denen die Menschen durch falsche 
Informationen in den sozialen Medien Hamsterkäufe
getätigt haben. Dies hat zu Engpässen in den
relevanten Produkten geführt.
Daduch wurde gezeigt, wie anfällig Lieferketten 
für drastische Veränderungen der Nachfrage sind.
Diese Fälle betrafen in erster Linie 
physische Güter wie Toilettenpapier.
Aber auch kritische Infrastrukturen, 
die für nicht-physische 
Güter wie Elektrizität oder Kommunikation 
benötigt werden, können ebenfalls von 
plötzlichen Nachfragespitzen betroffen sein. 
Kritische Infrastrukturen müssen gegenüber 
verschiedenen Arten von Bedrohungen widerstandsfähig sein.
Deswegen ist es wichtig, mögliche Bedrohungen 
für die Infrastruktur zu analysieren,
bevor diese eintreten, um Gegenmaßnahmen zu erarbeiten.
Daher werden in dieser Thesis die Auswirkungen 
von plötzlichen Nachfrageänderungen
im Kontext von Energiesystemen analysiert. 
Eine Möglichkeit, wie diese plötzlichen 
Nachfragespitzen entstehen, sind Gerüchte 
und Informationen, die über soziale 
Medien verbreitet werden.
Die Verbraucher erhalten Informationen, 
die sie dazu verleiten, dass sie 
ihr Konsumverhalten ändern. 
In dieser Arbeit werden die möglichen Auswirkungen 
von Nachfrageänderungen auf das Stromnetz 
aufgrund von Informationen im Internet modelliert und 
anhand verschiedener Szenarien analysiert. 
Dafür wurde ein Simulationsmodell implementiert.
Das Modell basiert auf einer graphbasierten 
Struktur zur Modellierung von Netzwerken in 
den sozialen Medien und einem epidemologischen 
Modell (welches auf dem Susceptible-Infectious-Recovered Modell, 
oder SIR, basiert))
zur Modellierung von Informationsverbreitungsprozessen. 
Außerdem wurden bestimmte Simulationsparameter 
anhand von realen Daten aus den sozialen Medien geschätzt.
Anschließend wurden vier theoretische 
Szenarien als Beispiele verwendet,
um sowohl die Simulationsumgebung als auch 
die Auswirkungen von plötzlichen
Nachfrageänderungen im Strommarkt zu analysieren. 
Die Analyse zeigte, dass die gleichzeitige Änderung
der Nachfrage einer signifikanten Anzahl von Konsumenten
ein Risiko für die 
Stabilität des Stromnetzes darstellen. 
Außerdem könnte die weitere Einführung von 
energieintensiven Technologien 
wie etwa Elektrofahrzeuge 
die Auswirkungen von synchronisierten Änderungen der 
Nachfrage auf die Energieinfrastruktur verstärken.
\end{abstract}