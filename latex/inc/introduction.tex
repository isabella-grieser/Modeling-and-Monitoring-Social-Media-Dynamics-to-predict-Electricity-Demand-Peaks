\chapter{Introduction}

This chapter aims to
show the significance of information and the subsequent human action based on it
in the context of electrical power systems.
Furthermore, the necessity to understand the possible consequences 
of human behavior on critical infrastructure is shown. 
First, the necessary context and motivation for this thesis is provided
(see section \ref{contextmotivation}).
Then, the problem that is considered in this thesis is described 
(see section \ref{problemstatement}). Afterward, the contribution that this
thesis proposes is presented (see section \ref{contribution}), 
followed by the outline of this work (see section \ref{outline}).

\section{Context and Motivation}
\label{contextmotivation}

During the COVID-19 pandemic, drastic changes in consumer behavior
showed how sensitive supply markets can be to changes in consumer demand.
For this reason, research in the field of changes in demand during crisis
started to gain interest. One reason for changes in consumer demand is 
social media usage. Through social media, consumers can gain new information
on possible supply chain issues and react based on the information received 
through these channels \cite{naeem2021social}.
The research for these drastic changes in consumer demand are mostly
done for regular, physical goods. But the explanations given in these 
works may not be translated well to critical infrastructure such as 
water, energy or communications systems.

However, resilience of critical systems is of importance for modern society.



\section{Problem Statement}
\label{problemstatement}
\section{Contribution}
\label{contribution}
\section{Outline}
\label{outline}

First, the necessary background necessary for this thesis is explained
(see section \ref{background})