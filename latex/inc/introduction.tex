\chapter{Introduction}

This chapter aims to show the significance of information and 
the subsequent human action based on it
in the context of the power grid infrastructure.
Furthermore, the necessity to understand the possible consequences 
of human behavior on critical infrastructure is shown. 
First, in Section \ref{contextmotivation}, the necessary context and motivation for this thesis is provided
Then, the problem that is considered in this thesis is described 
in Section \ref{problemstatement}. Afterward, the contribution that this
Thesis proposes is presented in Section \ref{contribution}, 
followed by the outline of this work in Section \ref{outline}.

\section{Context and Motivation}
\label{contextmotivation}

During the COVID-19 pandemic, drastic changes in consumer behavior
lead to massive supply issues, thus showing how sensitive supply markets 
can be to changes in consumer demand.
For this reason, research in the field of changes in demand during crisis
started to gain interest. One reason for changes in consumer demand is 
social media usage. Through social media, consumers can gain new information
and change their demand on certain goods based on the information received 
through these channels \cite{naeem2021social}.
The research for these drastic changes in consumer demand are usually
done for regular, physical goods. But the results shown in these 
research works may not translate well to critical infrastructure
with other characteristics as physical goods such as 
water, energy or communications systems.

One critical infrastructure that is specially important for 
our society is the power grid infrastructure.
There are no examples of drastic changes in consumer behavior leading
based on information on social media leading to supply issues or
even the breakdown of the power grid in real life. Nonetheless, there 
are still possible events that could happen where the demand for
electricity may be increased because of information on social media.
For example, a false notice of reduced prices for electricity
during peak demand hours may lead to increased demand during 
a time where the demand is already high. The additional demand
may increase the total demand to a level that the power grid
cannot support, thus leading to blackouts.
Another example would be that coordinated actions to reduce
energy, either in regards to combat climate change \cite{earthday}
or as a means of an attack against the state by conspiracy theorists,
lead to a non-negligible mismatch of supply and demand.

Furthermore, the resilience of the power grid
There are no real-life examples of social media information 
effecting the power grid drastically. Nonetheless, it is of vital importance 
to the resilience of the power grid to analyse possible situations that 
may critically affect the resilience of the power grid. Thus,
theoretical, but possible scenarios should be analysed before they
arise in real-life.

%Reihenfolge:
%erst sagen: ja es gab bis jetzt noch keine situation like this 
% mit stromnetz: ABER mögliche szenarien nennen und sagen "ja ist möglich"
%dann sagen dass resilience wichtig ist und man muss das
% am besten checken BEVOR die Situation tatäschlich auftritt
% dann sagen, dass bis jetzt nur theoretische works und man vllt auch 
% ein praktischeres framework will?

\section{Problem Statement}
\label{problemstatement}

% Resilience von critical systemen sehr wichtig
% Es gibt schon theoretische Literatur: ist aber nur theoretisch
% UNBEDINGT schon ein paar mögliche Szenarien nennen
Since critical infrastrucutre is of vital importance for society,
their 

\section{Contribution}
\label{contribution}
\section{Outline}
\label{outline}

First, the necessary background necessary for this thesis is explained
in Chapter \ref{background}. This Chapter includes an introduction
to the concept of panic buying and the relevance of social media 
to this phenomenom in Section \ref{panicbuying}. Then, in 
Section \ref{graphbasics}, the foundations of graphs are introduced,
how they can represent real networks and how graphs can be
generated randomly. Next, in Section \ref{informationdiffsection},
different algorithms to model information propagation in 
social networks are described. Last, in \ref{powerloadsection},
different variables that are connected with human behavior
introduced and their correlation with power consumption are 
explained.

In Chapter \ref{relatedworks}, the related works of relevance for 
this Thesis is introduced.

Next, in Chapter \ref{implementationall}, the framework to
analyse and predict possible power demand surges based on
information on social media is introduced.
First, the simulator is introduced in Section \ref{modelsocialnetwork}.
Then, in Section \ref{parameterestimationalgo}, 
an algorithm to estimate parameters to realistically simulate
the effects of information on social media is introduced and 
the all steps of the framework are explained.

Then, in Chapter \ref{results} .....