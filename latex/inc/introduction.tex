\chapter{Introduction}

This Chapter aims to show the significance of information on social media and 
the changes in human behavior based on it
in the context of the electrical grid.
First, in Section \ref{contextmotivation}, the necessary context and 
motivation for this thesis is provided.
Then, the problem that is considered in this thesis is described 
in Section \ref{problemstatement}. Afterward, the contribution of this
thesis is presented in Section \ref{contribution}, 
followed by the outline of this work in Section \ref{outline}.

\section{Context and Motivation}
\label{contextmotivation}

During the COVID-19 pandemic, severe changes in consumer behavior
lead to massive supply issues, thus showing how drastic supply markets 
react to changes in consumer demand.
Consequently, there has been a growing interest in researching 
how shifts in demand manifest during crises. 
One reason for changes in consumer demand is 
social media usage. Consumers can receive new information
and change their demand on certain goods based on the information received 
through social media \cite{naeem2021social}.
Current research mostly considers regular, physical 
goods for these demand changes.
However, the results shown in these 
works may not translate well to critical infrastructure,
such as water, energy or communications systems,
since they have characteristics that differ from physical goods.

The power grid is a critical infrastructure 
that is an important for our society.
There are yet no examples of drastic changes in consumer behavior
based on information of social media leading to supply issues or
even the breakdown of the power grid. Nonetheless, 
there are still possible scenarios where the demand for
electricity may increase due to information on social media.
Moreover, the consequences of such events would be critical for the
reliability of the power grid.
For example, a false notice of reduced prices 
could lead to an unexpected demand increase which 
can be critical at times of peak demand. 
The additional demand
may increase the total demand to a level that the power grid
cannot support, thus leading to blackouts.
Another example would be that coordinated actions to reduce
power, either in regards to combat climate change \cite{earthday}
or as an attack against the state by conspiracy theorists,
could lead to a sudden, non-negligible mismatch of supply and demand.


\section{Problem Statement}
\label{problemstatement}

To reduce the probability of blackouts, 
the resilience of the power grid is of vital importance.
Resilience describes the ability of a system 
to maintain critical services, to recover from threats
and to adapt the system to future hazards 
\cite{wells2022modeling}.
To assure the resilience of the power grid, a variety of possible 
scenarios which could negatively affect the power grid need
to be analyzed. Moreover, countermeasures need to be made for all possible scenarios.
Thus, it is reasonable to analyze theoretical scenarios that, even though
they never happened until the present time, could plausibly happen. This 
would allow policymakers to draw countermeasures to these possible scenarios
and allow for quick and decisive actions to be taken by relevant 
parties.

Until now, there are already works which analyze the possible effects
of information, specially misinformation, on the power
grid. However, these works were of theoretical nature and
did not consider a detection system to warn of a possible 
failure of the power grid. Furthermore, these works did mainly analyze the 
possibility of a false notice of reduced electricity prices.
Other scenarios were usually not considered.

\section{Contribution}
\label{contribution}
This thesis makes multiple contributions to addressing the issue of 
power demand fluctuations due to social media.
First, this thesis both introduces and 
analyzes multiple different scenarios, in which
information on social media leads to a change in power demand.
Second, this thesis proposes a framework to simulate and predict if 
a certain topic is propagating through social media
in a speed that could lead to issues for the power grid.

\section{Outline}
\label{outline}

First, the necessary background for this thesis is explained
in Chapter \ref{background}. 
In Chapter \ref{relatedworks}, works related to 
this thesis are introduced.
Next, in Chapter \ref{implementationall}, the framework to
analyze and predict possible power demand surges based on
information on social media is introduced. Furthermore,
an algorithm to estimate parameters to realistically simulate
the effects of information on social media is introduced and 
the all steps of the framework are explained.
In Chapter \ref{results}, the different scenarios 
considered in this thesis are introduced and analyzed. Both the 
framework implemented in Chapter \ref{implementationall}
and the general risk of such scenarios are 
examined.
Last, in Chapter \ref{conclusion}, the conclusion drawn 
from this thesis is described. Moreover, the possibilities
of future research associated with this thesis 
are mentioned.