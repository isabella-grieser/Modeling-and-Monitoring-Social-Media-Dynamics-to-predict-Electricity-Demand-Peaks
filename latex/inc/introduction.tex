\chapter{Introduction}

This chapter aims to show the significance of information and 
the subsequent human action based on it
in the context of the power grid infrastructure.
Furthermore, the necessity to understand the possible consequences 
of human behavior on critical infrastructure is shown. 
First, in Section \ref{contextmotivation}, the necessary context and motivation for this thesis is provided
Then, the problem that is considered in this thesis is described 
in Section \ref{problemstatement}. Afterward, the contribution that this
Thesis proposes is presented in Section \ref{contribution}, 
followed by the outline of this work in Section \ref{outline}.

\section{Context and Motivation}
\label{contextmotivation}

During the COVID-19 pandemic, drastic changes in consumer behavior
showed how sensitive supply markets can be to changes in consumer demand.
For this reason, research in the field of changes in demand during crisis
started to gain interest. One reason for changes in consumer demand is 
social media usage. Through social media, consumers can gain new information
and change their demand on certain goods based on the information received 
through these channels \cite{naeem2021social}.
The research for these drastic changes in consumer demand are mostly
done for regular, physical goods. But the explanations given in these 
research works may not translate well to critical infrastructure
with other characteristics as physical goods such as 
water, energy or communications systems.

However, resilience of critical systems is of importance for modern society.
%Punkte zu sagen:
% Resilience von critical systemen sehr wichtig
% Es gibt schon theoretische Literatur: ist aber nur theoretisch
% UNBEDINGT schon ein paar mögliche Szenarien nennen

%Reihenfolge:
%erst sagen: ja es gab bis jetzt noch keine situation like this 
% mit stromnetz: ABER mögliche szenarien nennen und sagen "ja ist möglich"
%dann sagen dass resilience wichtig ist und man muss das
% am besten checken BEVOR die Situation tatäschlich auftritt
% dann sagen, dass bis jetzt nur theoretische works und man vllt auch 
% ein praktischeres framework will?

\section{Problem Statement}
\label{problemstatement}
\section{Contribution}
\label{contribution}
\section{Outline}
\label{outline}

First, the necessary background necessary for this thesis is explained
(see section \ref{background})