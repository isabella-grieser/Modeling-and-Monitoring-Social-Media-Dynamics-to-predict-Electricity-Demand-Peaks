\chapter{Introduction}

The following chapter aims to
shows the significance of misinformation in the context of
electrical power systems and the necessity to understand its
possible consequences to critical infrastructure. 
First, the necessary background necessary for this thesis is explained
(see Section \ref{background})

An Introduction should contain the theoretical background, 
the problem adressed, the problem's implication in science and/or society. 
It should than explain a short summary of the state of the art  as well 
as the composition of the thesis.

\section{Context and Motivation}
Since their development, web-based technologies 
becoming more and more of great importance for modern societies.
These technologies can be used in a variety of fields and methods.
The electrical grid is one of the fields that profit off of these new 
technologies. 
% talk about digitalization of the electrical grid

However, the increased usage of web-based technologies both in the 
electrical infrastructure and in general society adds new, potential points
of attack to the electrical grid, thus compromising the resilience of the 
power grid

One of these points of attack are possible cybersecurity threats
%Talk about Power Load as a Power Outage Risk
\section{Problem Statement}
\section{Contribution}
\section{Outline}