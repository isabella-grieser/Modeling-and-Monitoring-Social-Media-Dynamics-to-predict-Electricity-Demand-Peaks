\chapter{Implementation of a Framework to model the Effect
of Information on Social Media on the Power Grid}

In this Chapter, a framework to analyse the effects of information on the 
power grid will be introduced and the specific implementation of the 
framework will be explained.
First, in section \ref{simulationframeworksection}, 
the implementation of the simulation framework will described.


\section{Implementation of the Simulation Framework}
\label{simulationframeworksection}
There are multiple components that are essential to model the effects of 
both true and false information on critical infrastructures
such as the power grid. The development of the simulation framework
proposed in this work can be divided into three parts. First, it is necessary to model an 
example social media network graph with characteristics similar to
real social media networks. Second, an algorithm to model the 
propagation of information over the network needs to be defined and 
implemented. Third, rules to estimate the changes in the power consumption 
need to be defined and a general simulation algorithm needs to be implemented.




\section{Extending the Framework to estimate the System Parameters based on
Domain Variables}

In Section \ref{simulationframeworksection}, a framework to simulate the effects 
of information on the power grid was introduced. However, the simulation results
may differ significantly based on which parameters were chosen for the simulation.
Furthermore, if the systems parameters are chosen without any real-life basis,
the result of the simulation may be unrealistic. Thus, the simulation framework 
may not be useful to predict possible crisis scenarios.
As a consequence, if the framework should produce realistic results and if
it should be used to possibly predict overconsumption trends, an 
algorithm to estimate realistic values for the system parameters are necessary.

In Section \ref{powerloadsection}, different possible variables that 
correlate with power load were introduced. These variables could be 
used to estimate realistic system parameters. For this Thesis, social media data is 
used to estimate the system parameters that control the 
information propagation.
To use social media data as a relevant input data source for the framework,
a method is proposed in this Thesis. The steps in the proposed method are made of:

\begin{enumerate}
    \item Find all relevant posts and messages on social media websites.
    This can be done with both natural language processing techniques and
    by using keyword search queries.
    \item Count the amount of relevant social media posts over a given 
    timespan. This is done by counting the amount of posts in given intervals, 
    such as e.g. the amount of posts each 15 minutes. This data shows the
    engagement on the specific topic over time.
    \item Filter the data graph to smoothen the data and to facilitate the
    parameter extraction process.
    \item Use the processed data to calculate the system parameters
    for the SIR model.
\end{enumerate}

To estimate the system parameters with the processed data, 
% HIER das konzept von minimize und solve diff equations erklären und
% cooles Bild rein wäre nice




To calculate the differential equations for the specific SIR model algorithm
used in this Thesis, mean field theory 
is used to generalize the equations %put here ref to equations and cite 
% mean field theory shit


First, it can be assumed that for a Barabási–Albert random graph
that was defined in Section \ref{randomgraphssection}, 
the median degree for a node in the graph will be equal to the amount 
of edges generated when adding a new node to the random graph.
This assumption can be made since the graph follows the power law distribution,
which defines that few nodes have a high amount of connections to other nodes
and a high amount of nodes have comparatively low amount of connections to 
other nodes. Since the minimum amount of edges that a node can have 
in the Barabási–Albert random graph generation algorithm is $k$, this 
means that we can assume that $\forall i \to c_i=k$.
Next, we define that $S(t), I(t), R(t)$, where $S(t)$ is the amount of
entities in the system with the status \textit{Susceptible} at the time $t$, 
$I(t)$ is the amount of entities with the status \textit{Infected} and
$R(t)$ is the amount of entities with the status \textit{Recovered}. 
We assume that the states of the neighbors of a node $i$ at the time $t$ are 
randomly distributed given the probabilities $p^S(t), p^I(t), p^R(t)$,
where $p^S(t) + p^I(t) + p^R(t) = 1$. We can calculate the values $S, I, R$
with the Equations in \ref{SIR-system-amound-eqs:all-lines},
with $N$ as the total amount of entities in the system.

\begin{subequations}
\begin{align}
    S(t) &=p^S(t)\cdot N \label{SIR-system-amound-eqs:1}\\
    I(t) &=p^I(t)\cdot N \label{SIR-system-amound-eqs:2}\\
    R(t) &=p^R(t)\cdot N \label{SIR-system-amound-eqs:3}
\end{align}
\label{SIR-system-amound-eqs:all-lines}
\end{subequations}

Given the Equations in \ref{SIR-system-amound-eqs:all-lines}, it is also possible 
to calculate the changes $\Delta S, \Delta I, \Delta R$ in the system.
The changes can be calculated with the Equations defined in 
\ref{SIR-diff-system-amount-eqs}.

\begin{subequations}
\begin{align}
    \Delta S &=\Delta p^S(t)\cdot N \label{SIR-diff-system-amount-eqs:1}\\
    \Delta I &=\Delta p^I(t)\cdot N \label{SIR-diff-system-amount-eqs:2}\\
    \Delta R &=\Delta p^R(t)\cdot N \label{SIR-diff-system-amount-eqs:3}
\end{align}
\label{SIR-diff-system-amount-eqs}
\end{subequations}

Since we assume that the states are randomly distributed over the system, 
we consequently assume that the state of the node $i$ is randomly distributed.
Thus, the state $s_i$ of the node $i$ can be defined as in Equation 
\ref{state-node-equation}.

\begin{equation}
    s_i = [s_i^S, s_i^I, s_i^R] = [p^S(t), p^I(t), p^R(t)]
    \label{state-node-equation}
\end{equation}

Furthermore, the functions $g_i,f_i$ defined in Equations aaa and bbbb
can be generalized by assuming that the average amount of neighbors 
of a specific state $K$ can be calculated as $k\cdot p^K(t)$. For this
assumption, $k$ is the median degree of all nodes and $p^K(t)$ is the 
average probability that a node is in the state $K$.
Thus, the functions $g_i,f_i$ can be generalized as in the
Equations \ref{generalized-functions-g-f}.

\begin{subequations}
\begin{align}
    f_i(t) &= \beta \frac{n_i^I(t)(1+\alpha)}{n_i^I(t)(1+\alpha)+n_i^R(t)(1-\alpha)} 
    \nonumber\\
    \to f(t) &= k\beta \frac{p^I(t)(1+\alpha)}{p^I(t)(1+\alpha)+p^R(t)(1-\alpha)}
    \label{generalized-function-f} \\
    g_i(t) &= \beta \frac{n_i^F(t)(1-\alpha)}{n_i^I(t)(1+\alpha)+n_i^R(t)(1-\alpha)} 
    \nonumber \\
    \to g(t) &= k\beta \frac{p^F(t)(1-\alpha)}{p^I(t)(1+\alpha)+p^R(t)(1-\alpha)}
    \label{generalized-function-g}
\end{align}
\label{generalized-functions-g-f}
\end{subequations}

With these assumptions, the general equations for the SIR model can 
be deduced. First, the function for $\Delta R(t)$ can be infered
as shown in Equation \ref{delta-r-deduction-eqs}.

\begin{align}
    p^R(t+1) &= g \cdot p^S(t) + p_{verify}\cdot p^I(t) + p^R(t) \nonumber\\
    p^R(t+1) - p^R(t) &= g \cdot p^S(t) + p_{verify}\cdot p^I(t) \nonumber\\
    \Delta R(t) = (p^R(t+1) - p^R(t))\cdot N 
    &= N(g \cdot p^S(t) + p_{verify}\cdot p^I(t)) \nonumber\\
    &= N(g \cdot \frac{S(t)}{N} + p_{verify}\cdot \frac{I}{N} ) \nonumber\\
    &= g \cdot S(t) + p_{verify}\cdot I(t) \nonumber\\
    &= k\beta \frac{p^R(t)(1-\alpha)}{p^I(t)(1+\alpha)+p^R(t)(1-\alpha)} 
    \cdot S(t) + p_{verify}\cdot I(t) \nonumber\\
    &= k\beta \frac{\frac{R(t)}{N}(1-\alpha)}{\frac{I(t)}{N}(1+\alpha)+\frac{R(t)}{N}(1-\alpha)} 
    \cdot S(t) + p_{verify}\cdot I(t) \nonumber\\
    \Delta R(t) &= \frac{k\beta}{N} \frac{R(t)(1-\alpha)}{I(t)(1+\alpha)+R(t)(1-\alpha)} 
    \cdot S(t) + p_{verify}\cdot I(t) \label{delta-r-deduction-eqs}
\end{align}

The equation for $\Delta I(t)$ can be infered in a similar manner to $\Delta R(t)$.
The deduction steps for $\Delta I(t)$ are shown in Equation 
\ref{delta-i-deduction-eqs}.

\begin{align}
    p^I(t+1) &= f \cdot p^S(t) + (1 - p_{verify})\cdot p^I(t) \nonumber\\
     &= f \cdot p^S(t) + p^I(t) - p_{verify}\cdot p^I(t) \nonumber\\
    p^I(t+1) - p^I(t) &= f \cdot p^S(t) - p_{verify}\cdot p^I(t) \nonumber\\
    \Delta I(t) = (p^I(t+1) - p^I(t)) \cdot N 
    &= N (f \cdot p^S(t) - p_{verify}\cdot p^I(t)) \nonumber\\
    &= N (f \cdot \frac{S(t)}{N}  - p_{verify}\cdot \frac{I(t)}{N}) \nonumber\\
    &= f \cdot S(t) - p_{verify}\cdot I(t) \nonumber\\
    &=  k\beta \frac{p^I(t)(1+\alpha)}{p^I(t)(1+\alpha)+p^R(t)(1-\alpha)}
     \cdot S(t) - p_{verify}\cdot I(t) \nonumber\\
    &=  k\beta \frac{\frac{B(t)}{N}(1+\alpha)}{\frac{I(t)}{N}(1+\alpha)+\frac{R(t)}{N}(1-\alpha)}
     \cdot S(t) - p_{verify}\cdot I(t) \nonumber\\
     \Delta I(t) &=  \frac{k\beta}{N} \frac{I(t)(1+\alpha)}{I(t)(1+\alpha)+R(t)(1-\alpha)}
     \cdot S(t) - p_{verify}\cdot I(t) \label{delta-i-deduction-eqs}
\end{align}

Last, given that there are no entities entering or leaving the system,
it can be assumed that the state changes in the system all sum to zero,
thus $\Delta S+ \Delta I+ \Delta R = 0$. With this, the equation for
$\Delta S$ can be deduced as shown in Equation \ref{delta-s-deduction-eqs}.

\begin{align}
    \Delta S &= - \Delta I - \Delta R \nonumber\\
     &= -\frac{k\beta}{N} \frac{I(t)(1+\alpha)}{B(t)(1+\alpha)+R(t)(1-\alpha)}
     \cdot S(t) + p_{verify}\cdot I(t) \nonumber\\
      & -\frac{k\beta}{N} \frac{R(t)(1-\alpha)}{B(t)(1+\alpha)+R(t)(1-\alpha)} 
      \cdot S(t) - p_{verify}\cdot I(t) \nonumber\\
      &= -\frac{k\beta}{N} \frac{I(t)(1+\alpha) + R(t)(1-\alpha)}{I(t)(1+\alpha)+R(t)(1-\alpha)}
      \cdot S(t) + p_{verify}\cdot I(t) - p_{verify}\cdot I(t) \nonumber\\
      &= -\frac{k\beta}{N} \frac{I(t)(1+\alpha) + R(t)(1-\alpha)}{I(t)(1+\alpha)+R(t)(1-\alpha)}
      \cdot S(t) \nonumber\\
      &= -\frac{k\beta}{N} \label{delta-s-deduction-eqs}
\end{align}
