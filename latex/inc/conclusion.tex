\chapter{Conclusion}
\label{conclusion}

The COVID-19 pandemic showed that drastic changes in 
consumer behavior can lead to significant supply chain issues.
As a consequence, there is a growing interest in researching 
how shifts in demand manifest during crises. 
Critical infrastructure such as the power grid
do not seem to have been yet affected by such demand changes.
Nonetheless, these demand shifts could affect the critical 
infrastructure in the future. In order to 
formulate countermeasures to such risks, it is 
important to analyze how these demand shifts could 
arise and how they would progress. In this thesis,
a framework was implemented to simulate the effects of 
changes in consumer behavior. The focus was set on
consumer behavior changes based on the spread
of information on social media. 
The simulation framework was implemented using 
a Barabási–Albert random graph as a representation for social 
media networks. Moreover, an epidemological model, 
the susceptible-infectious-recovered model, is used to model 
the information propagation in a social media network.
In addition, an algorithm was implemented to estimate 
simulation parameters based on real social media 
data. 

The simulation framework was used to analyze both 
the simulation framework and the effects of information 
on social media on consumer behavior and 
consequently on the power grid infrastructure.
For this, four different 
scenarios were introduced where information
on the internet could lead to consumer demand changes. 
The first scenario dealt with the concept of 
demand-response programs. The second focused on 
coordinated acts by a conspiracy theorists to destabilize 
the power grid. The third one dealt with rumors leading 
to a mass panic, thus leading people to charge their 
electric vehicles and leaving their homes in masses.
The last one comprises of an emergency message 
which requests people to shower immediately.
The analysis of the scenarios 
showed that syncronized consumer
demand changes pose a significant threat to the power
grid. The risk for the power grid depends on the percentage
of the population that is willing to change their behavior 
due to the information being spread.
Furthermore, the adoption
of energy-intensive technologies such as 
electric vehicles may increase the 
effects of syncronized consumer demand changes
on the infrastructure. 
Hence, the risk of excess power demand events may
elevate with increased adoption of new, energy consuming 
technologies in society.
The results show that policymakers should consider the
threat of misinformaion and formulate countermeasures
to deal with possible consequences of such events.

\section{Future Works}

The work described in this thesis can be extended in multiple
directions. First, more scenarios could be researched 
to analyze the effects of changes in consumer behavior
on the power grid further. Second, more real world 
data sources could be used in conjunction with the 
simulation framework.
In Section \ref{variabledependency}, different data 
sources were introduced and their relevance in regards 
to power load forecasting were explained. For this
thesis, only social media data was used to estimate system 
parameters. However, other data sources could be used
to either estimate simulation parameters or to predict
sudden changes in consumer demand. For example,
traffic data could be used to analyze movement flows 
between different parts of a city. Therefore, the
traffic flow out of residential districts into 
commercial and industrial districts of the city 
could be observed. With this,
the percentage of households with at least one 
resident at home could be estimated. 
This would allow for a more realistic estimation 
of which households could affect the power grid.
Third, the excess power consumption could be mapped
to a physical location. Consequently, with this, 
the power demand changes could be mapped to the 
physical power grid infrastructure. Thus, 
critical levels of demand changes can be 
checked with the local critical thresholds of the
local power grid connections. Social media data could
be used to estimate the possible location of the 
posting entities \cite{jurgens2015geolocation}.
However, other possible methods could be analyzed.
Forth, more detailed methods to estimate excess 
power consumption could be analyzed.
In Section \ref{loadprofilebasic}, the concept 
of load profiles were introduced. One 
method to generate load profiles is by 
creating a model which 
analyzes the power consumption of a single 
household and extrapolates the results to 
a group of households. The model includes 
the appliance usage in a household. 
Those models could be used to create more 
realistic excess power consumption data
by dynamically activating and deactivating
household appliances of the model when 
households are considered as 
\textit{Infected} in the SIR model.
Fifth, the process of gathering relevant social 
media posts could be analyzed. In this thesis,
it is assumed that the relevant social 
media posts are already known. However,
finding posts that are of importance for 
specific scenarios may be difficult and 
dependent on which specific scenario is 
occuring. Thus, further work in analyzing 
how to gather relevant posts is necessary.
Sixth, possible countermeasures for this type 
of risk could be proposed. In other related works,
load shedding was introduced as a method to 
deal with misinformation-based attacks on the power grid.
However, other possible countermeasures to deal with 
this kind of danger could be considered.

