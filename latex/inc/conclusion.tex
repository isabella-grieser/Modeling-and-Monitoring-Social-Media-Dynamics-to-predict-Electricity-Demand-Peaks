\chapter{Conclusion}
\label{conclusion}

The COVID-19 pandemic showed that drastic changes in 
consumer behavior can lead to significant supply chain issues.
As a consequence, there is a growing interest in researching 
how shifts in demand manifest during crises. 
Critical infrastructure such as water or power grid 
infrastructure was still not affected by such demand changes.
Nonetheless, these demand shifts could affect critical 
infrastructure points in the future. In order to 
formulate countermeasures to such risks, it is 
important to analyze how these demand shifts could 
manifest and how they would progress. In this Thesis,
a framework was implemented to simulate the effects of 
changes in consumer behavior. The focus was set on
consumer behavior changes based on the spread
of information on social media. 
The simulation framework was implemented using 
a generated random graph as a representation for social 
media networks. Moreover, a susceptible-infectious-recovered 
(SIR) model, an epidemological model, is used to model 
the information propagation in the social media network.
Entities considered as \textit{Infected}
by the SIR model use an additional amount of 
electricity. The excess power consumption is 
calculated on an appliance level.
In addition, an algorithm was implemented to estimate 
simulation parameters based on real social media 
data. Then, the simulation framework was
used to analyze both the simulation framework and
the effects of information 
on social media on consumer behavior and 
consequently on the power grid infrastructure.
 For this, four different 
scenarios were introduced where information
on the internet could lead to consumer demand
changes. 
The analysis of the scenarios 
shows that syncronized consumer
demand changes pose a real threat to the power
grid infrastructure. Furthermore, the adoption
of energy-intensive technologies such as heat
pumps or electric vehicles may increase the 
effects of syncronized consumer demand changes
on the infrastructure.

\section{Future Works}

The work described in this Thesis can be extended in multiple
directions. First, more scenarios could be researched 
to analyze the effects of changes in consumer behavior
on the power grid further. Second, more real world 
data sources could be used in conjunction with the 
simulation framework.
In Section \ref{variabledependency}, different data 
sources were introduced and their relevance in regards 
to power load forecasting were explained. For this
Thesis, only social media data was used to estimate system 
parameters. However, other data sources could be used
to either estimate simulation parameters or to predict
sudden changes in consumer demand. For example,
traffic data could be used to analyze movement flows 
between different parts of a city. Therefore, the
traffic flow out of residential districts into 
commercial and industrial districts of the city 
could be observed. With this, it could be estimated
the percentage of households with at least one 
resident who could change the demand for electricity.
Third, the excess power consumption could be mapped
to a physical location. Consequently, with this, 
the power demand changes could be mapped to the 
physical power grid infrastructure. Thus, 
critical levels of demand changes can be 
checked with the local critical thresholds of the
local power grid connections. Social media data could
be used to estimate the possible location of the 
posting entities \cite{jurgens2015geolocation}.
However, other possible methods could be analyzed.
Forth, more detailed methods to estimate excess 
power consumption could be analyzed.
In Section \ref{loadprofilebasic}, the concept 
of load profiles were introduced. One 
method to generate load profiles is by 
creating a model which 
analyzes the power consumption of a single 
household and extrapolates the results to 
a group of households. The analysis includes 
appliance usage in a household. 
Those models could be used to create more 
realistic excess power consumption data
by dynamically activating and deactivating
household appliances of the model when 
households are considered as 
\textit{Infected} in the SIR model.




%POINTS: more appliance usage more detailed with load profile models
% use more of the variables for power load forecasting
% specially mention using traffic data to estimate
% household occupation parameters
% mention the idea of geolocalizing the excess power consumption
% mention analyse more scenarios
