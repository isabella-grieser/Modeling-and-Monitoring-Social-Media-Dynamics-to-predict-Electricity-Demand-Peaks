\chapter{Background}
\label{background}

In this chapter, the fundamental concepts necessary for this thesis are explained.

\section{The Electrical Power Grid as a Critical Infrastructure}

\subsection{Power Load as a Power Outage Risk}

\subsection{Social Network-based Attacks on the Electrical Power Grid}

\section{Graph as a Representation of Real-World Systems}
\label{graphbasics}
A graph is an abstract structure that represents a set of objects and the relationship 
between the objects. Graphs can be used to model a variety of real-world systems.
Notable examples of systems that can be modelled as graphs that are of 
interest in this thesis are social networks 
\cite{socialgraphexample} and power grids \cite{powergraphexample}, but graphs can 
also be applied to other real life systems, such as economic 
or transportation systems \cite{economicsgraph}. %TODO: cite for transportation


Per definition, a graph $G(V, E)$ consists of a set of vertices $V$ and a 
set of edges $E$. Each edge connects two vertices. This means that for all $E$, the 
condition  $E \subseteq\{ \{x, y\} \mid x, y \in V  \}$ 
is valid.

There are multiple different subtypes of graphs. A directional graph is a graph 
whose edges have a source node $a$ and target node $b$, thus representing
a direction going from $a$ to $b$. If the edges do not fulfill this 
condition, thus if the edges do not show a direction, then a graph is unidirectional.
In addition, a graph may have attributes, so-called weights, associated to its edges.
These graphs are called weighted graphs.

There are also multiple attributes that can be calculated in a graph. 
Relevant attributes are:
\begin{itemize}
    \item The degree $deg(v)$ of a vertex $v \in V$ is the number of edges connected to the
    vertex $v$. 
    \item The average path length is the distance between two vertices is
    defined as the number of edges along the shortest path
    connecting them %TODO: verbessern
    \item Clustering coefficient: The clustering coefficient, , CCi
    of a vertex i
     is the ratio between the actual number of edges
    that exist between the vertex and its neighbors and the
    maximum number of possible edges between these
    neighbors. The
    CC
    of the network is defined %TODO: verbessern
\end{itemize}

\subsection{Modelling Social Networks as Graph Systems}


\subsection{Random Graphs}

Graphs do not need to be based on real systems. They can also be generated. 
These randomly generated graphs are called random graphs \cite{randomgraphs}.



\section{Misinformation in the Social Media Age}

In recent years, there has been widespread concern that 
misinformation on social media is creating real-life damage in 
many different sectors of society and reducing the trust in govermental
institutions. Thus, the topic of online misinformation has become an 
important research field in the academic community.

\subsection{Information Diffusion in Social Networks}

One method to research the consequences of misinformation over social media
is to understand how the information spreads over
social networks and analyse occuring patterns in the propagation process. 
The spread of information over a given network is called
information diffusion. 
The modelling of information diffusion can be divided into three components
\cite{reviewinformationdiffusion}: 

\begin{itemize}
    \item State: Entities in a system may be grouped together in a class which
    defines their behavior in the system. In the context of information diffusion,
    an entity can for example belong to the  \glqq Spreader \grqq{} class which spreads the information
    to specific other entities in the system. An entity could also belong to a
    \glqq Susceptible \grqq{} class that did not receive the information, 
    but would possibly be able to receive it.
    \item Model: The model defines how the entities interact which each other
    given their states. A model is often represented by a graph with the entities
    as nodes and connections between the entities as edges 
    (See section \ref{graphbasics}).
    \item Transition rates: The transition rates are the parameters used
    do define how the information is diffused in the model dynamically. 
\end{itemize}

The first two components are independent of specific simulation scenarios,
unlike the third component, thus the first two components should be further
analysed. Furthermore, there are a multitude of different ways 
to model information diffusion in social networks. Therefore, the most common
modelling methods will be further analysed in this chapter.

\subsubsection{Epidemiological Models}

Due to their similarity to the spread of infectious diseases, 
the spread of misinformation is often modelled the same as epidemiological models.
The simplest epidemiological model is the SI model, where entities in a system
can either belong to the state \glqq Susceptible\grqq{} ($S$) or 
\glqq Infected\grqq{} ($I$) and where the amount of total entities of a 
system can be calculated as $N=S+I$. The SI model is the basis for many
more specialised models where the states are more detailed to capture more 
aspects of real life behavior. A few models based on the SI model and their 
specific state are shown in Table \ref{SI-table}.

\begin{table}[ht!]
    \centering
    \begin{tabular}{|c | c |} 
     \hline
     Acronym & Classes  \\ 
     \hline
     SIR & Susceptible - Infected - Recovered  \\ 
     \hline
     SIS & Susceptible - Infected - Susceptible \\
     \hline
     SEI & Susceptible - Exposed - Infected \\
     \hline
     SIRS & Susceptible - Infected - Recovered - Susceptible \\
     \hline
     SHIR & Susceptible - Hesitant - Infected - Recovered \\
     \hline
     SCIR & Susceptible - Contacted - Infected - Recovered \\
     \hline
    \end{tabular}
    \caption{Example models building on the SI 
    model for the spread of misinformation \cite{reviewinformationdiffusion}}
    \label{SI-table}
\end{table}

Of specific interest for this thesis are the SIS and SIR models. 
For the SIR model, it is assumed that at the beginning, all entities in the 
system are in the state $S$, thus susceptible to an infection. In the context of 
misinformation diffusion this means that an entity does not know about the
misinformation that is being spread and is neutral towards it. A susceptible
entity may become infected and thus may change its state $S\to I$.
An infected entity knows and believes in the misinformation and it may
spread the misinformation to other entities in the system. After being infected,
an entity may learn that the information is fake. Therefore, it recovers 
from being infected and it changes its state $I\to R$. An entity in a 
recovered state cannot be infected again by misinformation and is thus immune.
The SIS model differs from the SIR model in that in the last state change,
the entity does not become immune, but \glqq forgets\grqq{} about the fact
that the information is fake, and becomes susceptible to misinformation again.
%TODO: finde Grundlagenbuch für states und modellchanges

The amount of entities belonging to each state may change 
after a time duration $\Delta t$. The equations to calculate the 
state changes, and thus the progress of the information diffusion, 
can be seen in Table \ref{SI-table-equations}.

\begin{table}[ht!]
    \centering
    \begin{tabular}{|c | c |} 
     \hline
     SIR & SIS  \\ 
     \hline
     & \\
     $\begin{aligned}
          \frac{dS}{dt} &= -\beta I S \\
          \frac{dI}{dt} &= \beta I S - \gamma I \\
          \frac{dR}{dt} &= \gamma I  
        \end{aligned}$
      &
      $\begin{aligned}
          \frac{dS}{dt} &= -\beta I S + \gamma I\\
          \frac{dI}{dt} &= \beta I S - \gamma I
        \end{aligned}$
       \\ 
       & \\
     \hline
    \end{tabular}
    \caption{Model equations for SIR and SIS model \cite{sirequation}}
    \label{SI-table-equations}
\end{table}

$\beta$ and $\gamma$ are the transition rates parameters of the model, where 
$\beta$ is the transmission rate and $\gamma$ the 
recovery rate. Moreover, the equations in Table \ref{SI-table-equations}
are ordinary differential equation that are used to calculate the amount
of entities for each state in the system, not to calculate in which state
each entity in the system is in. To analyse the information diffusion on 
an entity level, graph based diffusion models are needed.
There is no exact definition for the algorithm for the state changes in the
graph model, but generally, the probability, that an entity changes
its state from susceptible to infected $S \to I$, increases depending on
how many connected entities are already infected. In its simplest form, a
propagation algorithm may be defined as in Equation \ref{eqbasicpropagation} 
\cite{easypropagation}.

\begin{equation}
    I(a) = 1 (\sum\limits_{\substack{(b,a)\in E, \\ b \in V \cap I}}
    1(f^{rand}\geq \beta)>0) 
    \label{eqbasicpropagation}
\end{equation}

In the equation, the infection status $I(a)$ of each node $a$
in the graph is defined by if any neighboring node $b$ that is connected 
to $a$ and is part of the infection node set $I$, 
and thus infected, fullfills the condition $f^{rand}\geq \beta$,
where $f^{rand}$ is the random probability generation function and $1$ 
is a function that equals one if the specific condition is fulfilled and 
zero otherwise.

%TODO: more stuff to write

\subsubsection{Information Cascade Models}

Another way to model information diffusion in graphs is by viewing it as a 
sequential information propagation process. This assumption is made in
information cascade models. In this type of model, any node $n$ infected in the
iteration step $i$ may infect its connected neighbor $a$ with a probability $P_n(a)$
\cite{reviewinformationdiffusion}. All nodes infected by node $n$
can then infect their neighboring node in the next iteration step $i+1$
and the node $n$ becomes inactive and does not infect any neighbors anymore.
The equation for the propagation step can be seen in Equation 
\ref{eqbasicpropagationcascading}.
Information cascade models are mostly used for prediction and influence 
analysis purposes, and not to explain the collective behavior
of the system during the information diffusion process.

\begin{equation}
    I(a) = 1 (\sum\limits_{\substack{(b,a)\in E, \\ b \in V \cap I}}
    1(f^{rand}\geq P_n(a))>0) 
    \label{eqbasicpropagationcascading}
\end{equation}

There are several different subtypes of cascading models.
Some subtypes are described in \cite{diffusionbasics}:

\begin{itemize}
    \item Independent Cascading Model: In this type of model, the 
    probability $P_n(a)=p$ ist constant for each node $a$ and neighbor $n$.
    Thus, the probability function does not depend on the history 
    of the system and its information diffusion status.
    \item Decreasing Cascading Model: In this type of model, the probability
    function to activate the node $a$ decreases with each attempts of its 
    neighbors. This means that if a neighbor $n$ unsuccessfully tries to infect
    $a$ at iteration step $i$, then the probability that the neighbor $m$
    can sucessfully infect $a$ at step $i+t$ is smaller, thus $P_n(a)>P_m(a)$.
\end{itemize}

\subsubsection{Threshold Models}
A different method to analyse the diffusion process in graphs is by seeing the
propagation step as a process where each entity needs to overcome a 
specific threshold $\theta$ to become infected. More specifically, 
a certain amount of neighbors need to be infected for the node $a$ to become 
infected, too. Furthermore, contrary to the cascade models, a node can always 
infect its neighbors as long as the conditions are fulfilled in the threshold 
model. The general equation for the threshold model can be seen in Equation
\ref{eq:threshold}.

\begin{equation}
    I(a) = \sum\limits_{\substack{(b,a)\in E, \\ b \in V \cap I}}
    1 > \theta    
    \label{eq:threshold}
\end{equation}

Subtypes of this kind of diffusion model differ by the threshold parameter.
Some subtypes are described in \cite{diffusionbasics}:

\begin{itemize}
    \item Majority Threshold Model: In this type of model, a node $a$ becomes
    infected if the majority of its neighbors are infected, thus 
    $\theta = \frac{1}{2}deg(a)$
    \item Small Threshold Model: In this type of model, the threshold for that
    $a$ becomes infected is small. The advantage of this model is that 
    certain algorithms may be calculated faster with this type of model 
    \cite{diffusionbasics} 
    \item Unanimous Threshold Model: In this type of model, all neighbors 
    of a node $a$ need to be infected for $a$ to become infected, thus
    $\theta = deg(a)$.
\end{itemize}

\section{Variable Dependencies in the Context of Power Usage}

The power demand is always changing. The changes in demand depend on a multitude 
of factors and variables and can change drastically over the day and over the year.
These variables can be classified as either endogenous or exogenous variables.
Exogenous variables can be considered as independent variables 
whose values determined outside of the system. 
Endogenous variables, on the other side, 
are variables that are dependent of other variables in the 
system.

In the context of power load prediction, the most commonly 
used endogenous variable is the historical power load data.
For the exogenous variables, there are multiple types of 
variables that are often considered as possible input data \cite{exogenousdata}
\cite{exogenousdata2}.
The first type of variables are environmental variables like temperature, 
rainfall, humidity or wind power. A different type of variables are time data
such as the weekday, if a day is a holiday and the time of the day.
One more type of variables are socio-economic variables such as the
population size and growth, the exchange rate, the income level,
the gross domestic product or the different types and amount 
of consumers like agricultural, industrial or household consumers.
Another type of variables are building and occupancy related variables such
as the household appliance usage,
the number of persons or the number of bedrooms in a household.

The relevance of each type of variables depend heavily on it is
a short-, middle- or long-term power load forecasting. When shifting to more 
long-term predictions, the slower changing variables such as socio-economic
variables become more important than short-term variables such as weather data 
\cite{loadforecastingtimedependency2}\cite{loadforecastingtimedependency}.
The specific task definition is also relevant for the parameter selection.
Power load prediction models for residential buildings may benefit from 
building and occupancy related variables, but for a prediction model on a 
macroeconomic scale, socio-economic variables are more important.

\subsection{Exogenous Variables based on Human Behaviour}

Exogenous variables that are explicitly linked to human behaviour, 
such as social media usage ,traffic information, 
% ICT location, IoT information (such if a device is turned on or off), 
satellite image data or internet usage, are not often 
used for power load forecasting. 
But these are the variables that would be affected the most in a change of 
human behavior. Thus, these variables should be analysed in more detail 
for this thesis.

Social media data does not directly correlate with power demand, but it is
possible to extract information relevant for power demand prediction
from the content generated by it:
First, it is possible to analyse the spacial density of people by 
counting the amount of tweets tagged in a specific location of interest. 
In the work of \textit{Deng et al.},
it was shown that there is a significant
correlation between the amount of geotagged tweets
and the power consumption in a specific region \cite{twittergeoloccorr}.
As a possible explanation for this correlation, the paper assumes that 
an increased amount of human activity in a specific region leads a greater 
use of facilities such as heating or air conditioning in buildings or 
other types of power-consuming behavior.
For this reason, geotagged tweets were already used as input data for 
power load forecasting models 
\cite{twittergeolocforecasting} \cite{twittergeolocforecasting2}.
Second, it is also possible to analyse the content written in the 
social media posts. The content in social media posts can be 
analysed in two ways: 
It can either be analysed by searching for posts with a defined keyword or tag
or it can be further analysed by using natural language processing techniques. 
In natural language processing the raw text content 
(and possibly its metadata and attachements) of a post is analysed
in some form to extract the desired information. Possible analysis
methods would be to classify either a post is related to some energy-consuming
behavior or to count the number of energy-relevant keywords in the text.
For example, a natural language processing model was trained to 
classify specifically energy-related tweets \cite{energybert}.
The other possibility is to query the social media instance 
to find all posts that contain the defined keyword or tag.
This method is usually simpler than using more sophisticated natural 
language processing techniques, but increases the amount of false positive 
posts that are not relevant for the specific task.
For power load prediction tasks, only keyword search is used.
In the work of \textit{LaLone et al.}, they show that the amount of energy
related tweets during hurricane Katrina correlated with the amount of 
power outages at the same time \cite{poweroutagetwitter}.
Another paper uses twitter data with both keyword search and 
tweets that were classified with natural language processing techniques to
to find power outage regions \cite{twitterpoweroutagelighttime}.
In addition, another paper used topic mining techniques to find
events on twitter posts that correlate with energy consumption 
\cite{twittertopicevent}.
A data type that can be considered as partially related to social 
network data, since it also shows the interest that greater parts of 
the population have for a specific topic, are search engine indices.
The work of \textit{Wu et al.} shows that adding keyword search counts
can improve power load forecasting accuracy \cite{googlepowerforecast}.

General internet traffic data also correlates with power usage.
In the study of \textit{Morleya et al.}, they conclude that 
electricity consumed by information and communication devices
may contribute to increased peak electricity demand 
\cite{internettrafficenergycorrelation}.
%TODO: describe how much ICT uses
%TODO: find corellation between internet traffic and people 
%staying at home
Therefore, internet traffic data can be used as input data for 
power demand forecasting models \cite{electricityinternetforecast}. 

Studies show that with the increased adoption of electric vehicles (EV), 
there is an increased interdependency between the electric and 
transportation infrastructure \cite{interdependnytrafficenergy}. 
Therefore, traffic data may correlate with power load 
in two ways:
First, with the traffic movement in general, it is possible to follow the
movement patterns of larger group. As an example, in the morning,
traffic intensity increases since larger part of the population needs
to go to their workplace. Subsequently, this population group
returns back to their residences after finishing their work in the afternoon.
Thus, there is also an increased traffic intensity in the afternoon.
This traffic movement can be analysed by measuring the 
incoming and outgoing traffic volume at the main roads of the 
region of interest. The usage of this variable in power load
forecasting models gained more interest during the COVID-19 pandemic, 
where drastically changing mobility and power consumption 
patterns meant that new model inputs were necessary to consider changing
behavior patterns in the population 
\cite{covidtrafficpower} \cite{covidtrafficpower2}.
But this variable also correlates with the electricity consumption pattern
outside of pandemic periods. One paper successfully showed that
using traffic data as an input for power load
forecasting models generally leads to more accurate predictions, even before
the COVID-19 pandemic \cite{causalmodeltrafficelectricity}.
Second, with the increasing EV usage in the future, 
the total power demand increases since these types of vehicles 
need to be charged often. Power load forecasting methods that consider this
issue often focus on charging stations \cite{evcharchingstations}
\cite{evcharchingstations2}, but changes in household power consumption 
and the general charging behavior of households
were also the objective of some studies. \textit{Gerossier et al.}
analysed in their study the different charging patterns in households
and found four different charging patters in which its members charged at 
a similar time and duration \cite{gerossier2019modeling}.
In addition, there are also studies that work on
power load prediction models for individual household
charging forecasts \cite{skala2023interval}. 
A notable study was done by \textit{Arias et al.} that also considered the traffic patterns
and analysed the differences in EV charging behaviors between
commercial and residential districts when forecasting the power demand
created by EV charging \cite{arias2016electric}.

The last variable that depends on human behavior mentioned in this section are
satellite images. These kind of images can be used to find a variety 
of enviromental factors that are of relevance for the amount of power demand.
Satellite images are mostly used to forecast power generation of 
renewable resources such as solar power \cite{solarprediction}.
But it can also be used to estimate household power consumption,
specially during nighttime. This is specially of relevance for regions 
where the electricity data is unreliable, such as in developing contries
\cite{reviewnighttime}. There is a relation 
between nighttime light intensity and power consumption.
But this relation depends on the type of consumer 
(households, industry, ...). Furthermore, the relation is not linear
\cite{nighttimepowerestimation}. 
A disadvantage of using nighttime images for power demand prediction is 
the inherent noise in the data. Non-household light sources such as 
streetlights and vehicles add noise to the image data \cite{reviewnighttime}.
Another usage of nighttime satellite image is for power outage detection.
If there is a power outage, the affected region will turn dark, which 
consequently can be detected in satellite images. Thus, sattelite image data 
was successfully used in multiple power outage detection models
\cite{nightpoweroutage} \cite{twitterpoweroutagelighttime}.
