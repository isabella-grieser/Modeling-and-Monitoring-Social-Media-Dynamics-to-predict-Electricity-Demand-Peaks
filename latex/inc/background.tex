\chapter{Background}
\label{background}

In this chapter, the fundamental concepts necessary for this thesis are explained.

\section{The Electrical Power Grid as a Critical Infrastructure}

\section{Misinformation in the Social Media Age}

\section{Graph as a Representation of Real-World Systems}

A graph is an abstract structure that represents a set of objects and the relationship 
between the objects. Graphs can be used to model a variety of real-world systems.
Notable examples of systems that can be modelled as graphs that are of 
interest in this thesis are social networks 
\cite{socialgraphexample} and power grids \cite{powergraphexample}, but graphs can 
also be applied to other real life systems, such as economic systems \cite{economicsgraph} 
or traffic .

Per definition, a graph $G(V, E)$ consists of a set of vertices $V$ and a 
set of edges $E$. Each edge connects two vertices. This means that for all $E$, the 
condition  $E \subseteq\{ \{x, y\} \mid x, y \in V  \}$ 
is valid.

There are multiple different subtypes of graphs. A directional graph is a graph 
whose edges have a source node $a$ and target node $b$, thus representing
a direction going from $a$ to $b$. If the edges do not fulfill this 
condition, thus if the edges do not show a direction, then a graph is unidirectional.
In addition, a graph may have attributes, so-called weights, associated to its edges.
These graphs are called weighted graphs.

There are also multiple attributes that can be calculated in a graph. The 
degree $deg(v)$ of a vertex $v \in V$ is the number of edges connected to the
vertex $v$.


Graphs do not need to be based on real systems. They can also be generated. 
These randomly generated graphs are called random graphs \cite{randomgraphs}.

\section{Information Diffusion in Social Networks}