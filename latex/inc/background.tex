\chapter{Background}
\label{background}

In this chapter, the fundamental concepts necessary for this thesis are explained.

\section{The Electrical Power Grid as a Critical Infrastructure}

\section{Misinformation in the Social Media Age}

In recent years, there has been widespread concern that 
misinformation on social media is creating real-life damage in 
many different sectors of society and reducing the trust in govermental
institutions. Thus, the topic of online misinformation has become an 
important research field in the academic community.



\section{Graph as a Representation of Real-World Systems}

A graph is an abstract structure that represents a set of objects and the relationship 
between the objects. Graphs can be used to model a variety of real-world systems.
Notable examples of systems that can be modelled as graphs that are of 
interest in this thesis are social networks 
\cite{socialgraphexample} and power grids \cite{powergraphexample}, but graphs can 
also be applied to other real life systems, such as economic systems \cite{economicsgraph} 
or traffic .

Per definition, a graph $G(V, E)$ consists of a set of vertices $V$ and a 
set of edges $E$. Each edge connects two vertices. This means that for all $E$, the 
condition  $E \subseteq\{ \{x, y\} \mid x, y \in V  \}$ 
is valid.

There are multiple different subtypes of graphs. A directional graph is a graph 
whose edges have a source node $a$ and target node $b$, thus representing
a direction going from $a$ to $b$. If the edges do not fulfill this 
condition, thus if the edges do not show a direction, then a graph is unidirectional.
In addition, a graph may have attributes, so-called weights, associated to its edges.
These graphs are called weighted graphs.

There are also multiple attributes that can be calculated in a graph. 
Relevant attributes are:
\begin{itemize}
    \item The degree $deg(v)$ of a vertex $v \in V$ is the number of edges connected to the
    vertex $v$. 
    \item The average path length is the distance between two vertices is
    defined as the number of edges along the shortest path
    connecting them %TODO: verbessern
    \item Clustering coefficient: The clustering coefficient, , CCi
    of a vertex i
     is the ratio between the actual number of edges
    that exist between the vertex and its neighbors and the
    maximum number of possible edges between these
    neighbors. The
    CC
    of the network is defined %TODO: verbessern
\end{itemize}

\subsection{Random Graphs}

Graphs do not need to be based on real systems. They can also be generated. 
These randomly generated graphs are called random graphs \cite{randomgraphs}.



\subsection{Information Diffusion in Social Networks}


\section{Variable Dependencies in the Context of Power Usage}

The power demand is always changing. The changes in demand depend on a multitude 
of factors and variables and can change drastically over the day and over the year.
These variables can be classified as either endogenous or exogenous variables.
Exogenous variables can be considered as independent variables 
whose values determined outside of the system. 
Endogenous variables, on the other side, 
are variables that are dependent of other variables in the 
system.

In the context of power load prediction, the most commonly 
used endogenous variable is the historical power load data.
For the exogenous variables, there are multiple types of 
variables that are often considered as possible input data \cite{exogenousdata}
\cite{exogenousdata2}.
The first type of variables are environmental variables like temperature, 
rainfall, humidity or wind power. A different type of variables are time data
such as the weekday, if a day is a holiday and the time of the day.
One more type of variables are socio-economic variables such as the
population size and growth, the exchange rate, the income level,
the gross domestic product or the different types and amount 
of consumers like agricultural, industrial or household consumers.
Another type of variables are building and occupancy related variables such
as the household appliance usage,
the number of persons or the number of bedrooms in a household.

The relevance of each type of variables depend heavily on it is
a short-, middle- or long-term power load forecasting. When shifting to more 
long-term predictions, the slower changing variables such as socio-economic
variables become more important than short-term variables such as weather data 
\cite{loadforecastingtimedependency2}\cite{loadforecastingtimedependency}.
The specific task definition is also relevant for the parameter selection.
Power load prediction models for residential buildings may benefit from 
building and occupancy related variables, but for a prediction model on a 
macroeconomic scale, socio-economic variables are more important.

Exogenous variables that are explicitly linked to human behaviour, 
such as traffic information, ICT location, IoT information (such if
a device is turned on or off), satellite image data (to check luminosity
of household regions) or 
internet or social media usage, are not often used. 
But these are the variables that would be affected the most in a change of 
human behavior.

Social media data does not directly correlate with power demand, but it is
possible to extract information relevant for power demand prediction
from the content generated by it:
First, it is possible to analyse the spacial density of people by 
counting the amount of tweets tagged in a location of interest. 
In the work of \textit{Deng et al.},
it was shown that there is a significant
correlation of 0.8 between the amount of geotagged tweets
and the power consumption in the specific region \cite{twittergeoloccorr}.
As a possible explanation for this correlation, the paper assumes that 
an increased amount of human activity in a specific region leads a greater 
use of facilities such as heating or air conditioning in buildings or 
other types of power-consuming behavior.
For this reason, geotagged tweets were already used as input data for 
power load forecasting models 
\cite{twittergeolocforecasting} \cite{twittergeolocforecasting2}.
Second, it is also possible to analyse the content written in the 
social media posts. The content in social media posts can be 
analysed in two ways: 
It can either be analysed by searching for posts with a defined keyword or tag
or it can be further analysed by using natural language processing techniques. 
In natural language processing the raw text content 
(and possibly its metadata and attachements) of a post is analysed
in some form to extract the desired information. Possible analysis
methods would be to classify either a post is related to some energy-consuming
behavior or to count the number of energy-relevant keywords in the text.
For example, a natural language processing model was trained to 
classify specifically energy-related tweets \cite{energybert}.
The other possibility is to query the social media instance 
to find all posts that contain the defined keyword or tag.
This method is usually simpler than using more sophisticated natural 
language processing techniques, but increases the amount of false positive 
posts that are not relevant for the specific task.
For power load prediction tasks, only keyword search is used.
In the work of \textit{LaLone et al.}, they show that the amount of energy
related tweets during hurricane Katrina correlated with the amount of 
power outages at the same time \cite{poweroutagetwitter}.
Another paper uses twitter data with both keyword search and 
tweets that were classified with natural language processing techniques to
to find power outage regions \cite{twitterpoweroutagelighttime}.
In addition, another paper used topic mining techniques to find
events on twitter posts that correlate with energy consumption 
\cite{twittertopicevent}.
A data type that can be considered as partially related to social 
network data, since it also shows the interest that greater parts of 
the population have for a specific topic, are search engine indices.
The work of \textit{Wu et al.} shows that adding keyword search counts
can improve power load forecasting accuracy \cite{googlepowerforecast}.

General internet traffic data also correlates with power usage.
In the study of \textit{Morleya et al.}, they conclude that 
electricity consumed by information and communication devices
may contribute to increased peak electricity demand 
\cite{internettrafficenergycorrelation}.
%TODO: describe how much ICT uses
%TODO: find corellation between internet traffic and people 
%staying at home
Therefore, internet traffic data can be used as input data for 
power demand forecasting models \cite{electricityinternetforecast}. 

Traffic data may correlate with power load in two ways: First, with the
increased adoption of electric vehicles (EV), the total power demand increases 
since these types of vehicles need to be charged often. 
%TODO: talk about works in this field
Second, with the traffic movement, it is possible to follow the
movement patterns of larger group. As an example, in the morning,
traffic intensity increases since larger part of the population needs
to go to their workplace. Subsequently, this population group
returns back to their residences after finishing their work in the afternoon.
Thus, there is also an increased traffic intensity in the afternoon 
%TODO:CITATION
