\chapter{Case Studies}

%do the usual introduction
% then explain wtf the idea is



\section{Scenario 1: Demand-Response Misinformation Campaign}
\label{demandresponsesection}
%First: Explain throuhout the scenario
%Then define assumptions
%Then define parameters
%Then simulation results
%This Scenario should also be the one with the total framework

With the increased usage of information and communication 
technologies (ICT), new methods of influencing
consumer demand patterns are possible. 
These can be used by electrical companies to change the 
demand curve over time, ideally to reduce peak loads 
and thus increasing the efficiency of the electrical grid.
A realistic power demand, as shown in Figure \ref{duckcurve}, tend 
to vary greatly during the day. This leads to inefficiency,
since excess supply sources need to be build to satisfy customers
during peak hours. The idea is to lead consumers to shift
their consumption from peak demand hours to low demand hours,
thus reducing the variation of demand over time and allowing
for more efficient management of the power grid infrastructure.
This concept of changing the demand to match the 
available supply of electricity is called demand-response. 
One method to influence the consumer demand is by real time 
pricing (RTP). With RTP, the electricity prices change
dynamically based on the total electricity demand.
The price changes are then forwarded to the customers with ICTs, 
who are thus informed of the price changes.
The idea of RTP is that consumers change their electricity demand
by shifting activities with a high electricity usage, 
such as washing clothes or cooking, to time periods 
where the total electricity demand and thus the prices
are lower. This leads to reduced peak energy consumption. 
RTP was already used in a variety of programs
under real life conditions \cite{barbose2004survey}.
Studies showed the benefits of such programs \cite{albadi2008summary}.

\begin{figure}[!ht]
    \center
    \includegraphics[scale=.75]{figs/duckcurve.png}
    \caption{real power load and the ideal power load for the 
    power grid infrastructure}
    \label{duckcurve}
\end{figure}

Problematic is if a false pricing rumor spreads through social media.
In 2022, changes in the verification policies of Twitter lead to users being able 
to impersonate well-known brands such as Pepsi, leading to Twitter 
users being unable to determine which social media channels were
posting trustworthy information \cite{twitterchaos}. One user could
have used the policy changes of Twitter to impersonate 	utility companies
and to advertise a false reduction of energy prices, leading to customers
believing the information and increasing their energy consumption as a 
reaction. Another possibility would be that hackers would force access to the
social media profiles of utility companies and start sending false 
information to its customers. Hackers were already able to gain access
to the social media profiles of multinational companies in the past
\cite{twitterhacker}. Hackers could also spread misinformation in regards
to a false reduction in prices, thus leading to changing consumer demand.

Based on the information, a possible scenario could be that hackers would
access the social media profiles of a well-known utility company
to spread the information that the energy prices would be reduced by 
a signifiant factor at a specific time. An example tweet is shown in 
Figure \ref{demandtweet}. This information would then be 
spread around social media, reaching many customers of the company.

\begin{figure}[!ht]
    \center
    \includegraphics[scale=.4]{figs/eondemandresponse.png}
    \caption{Example tweet posted by hackers pretending to be an utility company}
    \label{demandtweet}
\end{figure}

Given the description of the scenario, multiple assumptions can be made.
First, we can assume that the time where an entity receives the information
about the price reduction does not correlate with the time it will act on
it. Furthermore, since it is a widespread information propagation scenario,
we can use a similar information propagation event to analyze the 
results of the total framework. Thus, the information propagation
parameters do not need to be defined. Furthermore, one work estimates 
the price elasticity of electricity in Germany to be $-0.4310$
\cite{priceelasticity}. 
This means that a price reduction of $1\%$
on the electricity price would could lead to a $0.43\%$ increase in
electricity consumption. From this, we can assume that consumers 
respond well to such price reduction programs. Thus, we can also assume 
that many people will act on the information.

To estimate the information propagation parameters with the framework,
example social media data is necessary. However, there are no real examples
for social media information spread events which affected the power demand.
Thus, other information spreading events are used to estimate 
realistic propagation parameters. The dataset used for this scenario
was the Typhoon Haiyan dataset \cite{david2016tweeting}. In this
dataset, $2.753.230$ twitter posts and their published dates were collected
to analyze communication during a disaster. The twitter posts were collected
between November 8, 2013 and November 26, 2013. This means that the 
information propagation for this dataset progressed over multiple days.
The propagation curve and the estimated infection curves can be seen in 
Figure \ref{propagationestimationtyphoon}. In the figure, it can be seen 
that the parameter estimation algorithm models the infection progress
well, ignoring outliers such as the maximum value of the true information
propagation progress. Furthermore, the
information is assumed by the framework to have a high credibility 
of $\alpha=1$. The transmission rate is estimated to be average with 
$\beta=0.65$.
Since the infection process is slow and progresses over 
multiple days, the verification parameter is
low with $p_{\mathrm{verify}}=0.001$. Last, the degree-to-node ratio
is also low with $\frac{k'}{N'}=0.006$.


\begin{figure}[!ht]
    \center
    \includegraphics[scale=.95]{figs/parameter_estimation.png}
    \caption{Information propagation progress for the Typhoon dataset 
    and estimated infection process}
    \label{propagationestimationtyphoon}
\end{figure}

Besides the propagation parameters, which are estimated by the framework,
the system parameters $p_{\mathrm{will\_act}}$ and 
$p_{\mathrm{power\_usage}}$ need to be defined. Since the information 
propagated in the scenario
is believeable and we assume that customers respond well to price 
reduction programs, we can also assume that the two probabilities are 
reasonably high. For this simulation, values of $p_{\mathrm{will\_act}}=0.8$ and 
$p_{\mathrm{power\_usage}}=0.8$ were chosen. Furthermore, this scenario
is given for only one utility company and not for the whole 
energy market. We assume that $50\%$ of the households of a city belong
to an utility company X. Last, since the scenario would both be 
well-received by the entities and it is a believeable scenario, 
entities may also share the information with its neighbors even if
cannot act on the information. A reason for why an entity cannot act 
on the information is that it is the customer of another utility company.
This scenario considers the household appliances which are energy-intensive.
The config file of the scenario with all appliances considered in this
scenario can be found in the Appendix in Listing \ref{scenario1config}.

% next ideas: average over multiple seeds
% check out how unstable the simulation is
% then find out how the simulation acts based on the other two probabilities

To evaluate this scenario, the simulation was run five times with five 
different seeds used for the random probability calculations to ensure
different results. In addition, the simulation generates a social media
graph model with $1000$ nodes. The results of the simulation 
can be seen in Figure \ref{firstscenariobasicresult}.
The left plot shows the power consumption of the model. The right 
plot shows the infection progress during the simulation.
The lines in both plots show the average values over all simulations and 
the lighter-colored areas around the lines show the minimum and maximum values
over all simulations. 


\begin{figure}[!ht]
    \center
    \includegraphics[scale=.5]{figs/eval/scenario1/basic_run.png}
    \caption{Simulation results for the first scenario}
    \label{firstscenariobasicresult}
\end{figure}

For the power consumption curve in Figure \ref{firstscenariobasicresult}, 
it is noticeable
that the power consumption spike generated by the infection is directly
after the time where entities can reasoably start acting on the information.
In addition, few hours after the spike, the excess power consumption
reduces to an amount that the infractructure can handle. 
The reason is that even though there are still many infected entities
in the system, the duration in which specific household appliances
such as dishwashers run are limited. Thus, after running the specific
appliances, they do not consume any more power. Furthermore, the 
results of the framework are robust and show little variance over the
different simulations. 
Infection curve in the right plot shows 




\section{Scenario 2: Coordinated Action of Extremists 
against the Electrical Grid}

Echo chambers on various social media websites allow people to 
hear biased opinions and news which confirm their views on 
various topics \cite{terren2021echo}. This leads to political
polarization and extremism \cite{van2022banality}.
These people in turn can try to fight the established political
system. Research shows that social media influence people
to commit hate crimes \cite{muller2021fanning}.
In general, the number of violent incidents caused by
extremists, specially right-wing extremists, is rising 
\cite{koehler2016right}. 
Most of the criminal acts caused by political extremism 
in germany are in the fields of property damage, 
propaganda crimes, insults and hate speech \cite{bmicrimestatistics}.
But extremists also cause more serious crimes. In 2022,
right-wing extremists planned to sabotage the electrical grid 
infrastructure by destroying power lines \cite{anschlagstrom}.
Thus, it can be seen that critical infrastructure may also be a target
for political extremists. 

Another way to target critical infrastructure is with a coordinated
actions which in its sum can damage the infrastructure. The 
reasons for a group of people to take such action may be harmless,
such as the Earth day, where people where encouraged to switch off their 
lights and appliances to save energy \cite{earthday}.
But extremists could also synchronize their actions to reduce their
power consumption to a minimum, thus mismatching electricity supply
and demand by a considerable degree. Electricity companies would 
then need to deal with a sudden surplus of electricity.

Given the previous information, a possible scenario could be 
that extremists plan to turn off all their electrical devices at 
a specific time to destabilize the electrical infrastructure. The
time would be announced in advance. An example message spread by
these extremists can be seen in Figure \ref{schwurbler}.

\begin{figure}[!ht]
    \center
    \includegraphics[scale=.7]{figs/schwurblerchat.png}
    \caption{Example message posted by extremists on Telegram}
    \label{schwurbler}
\end{figure}

Given the description of the example scenario, multiple assumptions can be drawn.
First, this scenario also assumes that the time where an entity receives the
information does not equal to the time where it acts on the information.
Furthermore, this scenario is unlikely to make big parts of the population
to act on the information since it is assumed that only a small minority
of the population will act on extremist information.

\section{Scenario 3: Mass Evacuation of a City due to a Disaster}

Sometimes the conditions under a disaster get so extreme that 
authorities choose to order the evacuation of the affected population.
For these types of evacuation, the autorities may use emergency channels 
to inform the population of the evacuation.
But people may also choose to leave the city in masses on their own accord.
The information of a possible disaster, such as a wildfire, reaching the
city soon or other disasters such as a widespread terrorist attacks 
in the city.
If people decide to leave their homes to flee from the disaster, they 
will mostly use their cars as the mode of transportation.
In the future, the increased adoption of electric vehicles (EV)
may lead to people needing to recharge their EVs if they wish 
to drive them. An evacuation could lead to many people charging
their cars at the same time, thus creating excess demand that 
the infrastructure is unable to handle. 

Given the previous assumptions, a possible scenario could be that 
rumors about a wildfire spreading towards a city leads to people
trying to leave the city in masses. In addition, a significant 
amount of people drive EVs, thus giving them the need to charge
their cars before they are able to leave the city. An example
tweet spreading the rumor can be seen in Figure \ref{firetweet}.


\begin{figure}[!ht]
    \center
    \includegraphics[scale=.4]{figs/firenews.png}
    \caption{Example tweet posted by an anonymus user on Twitter}
    \label{firetweet}
\end{figure}

Given the example scenario, multiple assumptions for the simulation can
be made. First, it can be assumed that the time where an entity receives
the information equals the time where the entity will act on it.
Next, It can be assumed that the information will be shared very quickly.
Furthermore, it is assumed that most people will act on the information,
since they will fear for their lives.


\section{Scenario 4: Mass Showering after a Chemical Accident}
%This Scenario should also be the one with the total framework

With the increasing usage of ICTs in the population, it allows for 
new methods to communicate with people about extreme circumstances
such as natural disasters or other catastrophes. In 2022,
a new warning system called Cell Broadcast was introduced in Germany
\cite{techrichtlinie}. Cell Broadcast can be used to warn the population
of emergencies by sending all cellphone users in the 
region of interest a message with a loud alert tone.

Germany has one of the biggest chemical industry sectors in the world.
It is home to big chemical factories in cities such as Ludwigshafen and
Darmstadt. Areas with chemical factories have the risk of being
victims of chemical accidents.

A possible scenario would be that a chemical accident in a 
factory near a major city would lead to dangerous chemicals 
leaking out. These chemicals would spread through the air 
and adhere to the skin of humans, leading to serious 
health problems if they stayed on the skin for too long.
Thus, the city's government would use Cell Broadcast to
instruct the population to close their windows and 
to take a shower to remove the chemicals from their bodies.
Therefore, most people would take showers at the same time.
An example message can be seen in Figure \ref{warningmessage}.

\begin{figure}[!ht]
    \center
    \includegraphics[scale=.7]{figs/emergencychemical.png}
    \caption{Example warning message received via Cell Broadcast}
    \label{warningmessage}
\end{figure}

Given the scenario description, multiple assumptions can be
made for the simulation.
First, it is assumed that since Cell Broadcast is used, the simulation
will reach all entities in the city at the same time.
Furthermore, since the information comes from a verified channel,
all entities will act on the information. Also, the time where the
entities receive the information is also the time where they will act
on it.
Moreover, since this scenario is a synchronous event, we can define that the
maximum number of infected individuals at the same time equals
the total number of entities in the system, thus $I(t_{max})=N$.
Consequently, the usage of the simulation work is not necessary
and the results of this scenario can be calculated manually.
Darmstadt is chosen as the example city for this scenario analysis.

Darmstadt is a city with a population of $162.243$ in 2022 
\cite{statistadarmstadt}.  


% Questions: how to define the parameters
% For some, I can use a dataset and use this as an analogy
% But e.g. scenario 2 it doesnt work well
% how can I put thing out of my ass
% what could be interesting parameter tweakings 